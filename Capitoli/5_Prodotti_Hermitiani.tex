%\documentclass[a4paper,12pt]{article}
\usepackage[a4paper, top=2cm,bottom=2cm,right=2cm,left=2cm]{geometry}

\usepackage{bm,xcolor,mathdots,latexsym,amsfonts,amsthm,amsmath,
					mathrsfs,graphicx,cancel,tikz-cd,hyperref,booktabs,caption,amssymb}
\hypersetup{colorlinks=true,linkcolor=blue}
\usepackage[italian]{babel}
\usepackage[T1]{fontenc}
\usepackage[utf8]{inputenc}


\newcommand{\N}{\mathbb{N}} %naturali 
\newcommand{\R}{\mathbb{R}} %Reali
\newcommand{\K}{\mathbb{K}} % Campo K 
\newcommand{\C}{\mathbb{C}}
\newcommand{\F}{\mathbb{F}}

\newcommand{\B}{\mathfrak{B}} %Base B
\newcommand{\D}{\mathfrak{D}}%Base D
\newcommand{\RR}{\mathfrak{R}}%Base R 
\newcommand{\Can}{\mathfrak{C}}%Base canonica
\newcommand{\Rif}{\mathfrak{R}}%Riferimento affine
\newcommand{\AB}{M_\D ^\B }% matrice applicazione rispetto alla base B e D 

\newcommand{\vett}{\overrightarrow}
\newcommand{\sd}{\sim_{SD}}%relazione sx dx
\newcommand{\sbarra}{\backslash} %% \ 
\newcommand{\ds}{\displaystyle} 
\newcommand{\alla}{^}  
\newcommand{\implica}{\Rightarrow}
\newcommand{\ses}{\Leftrightarrow} %se e solo se
\newcommand{\tc}{\quad \text{ t. c .} \quad } % tale che 
\newcommand{\spazio}{\vspace{0.5 cm}}
\newcommand{\bbianco}{\textcolor{white}{,}}
\newcommand{\bianco}{\textcolor{white}{,} \\}% per andare a capo dopo 																					definizioni teoremi ...
\newcommand{\nvett}{v_1, \, \dots , \, v_n} % v1 ... vn
\newcommand{\ncomb}{a_1 v_1 + \dots + a_n v_n} %a1 v1 + ... +an vn
\newcommand{\nrif}{P_1, \cdots , P_n} 
\newcommand{\bidu}{\left( V^\star \right)^\star}

%per creare teoremi, dimostrazioni ... 
\theoremstyle{plain}
\newtheorem{thm}{Teorema}[section] 
\newtheorem{ese}[thm]{Esempio} 
\newtheorem{ex}[thm]{Esercizio} 

\newtheorem{cor}[thm]{Corollario} 
\newtheorem{lem}[thm]{Lemma} 
\newtheorem{al}[thm]{Algoritmo}
\newtheorem{prop}[thm]{Proposizione} 
\theoremstyle{definition} 
\newtheorem{defn}{Definizione}[section] 
\theoremstyle{remark} 
\newtheorem{oss}{Osservazione} 


% diagrammi commutativi tikzcd
% per leggere la documentazione texdoc

%\begin{document}
\section{Prodotti Hermitiani}
Ricordiamo la definizione di applicazione lineare
\begin{defn} [Lineare]\bianco
Sia $V$ un $\K$-spazio vettoriale.\\
$f:\, V \to V $ c \'e lineare se 
$$ \forall x,y \in V \quad \forall a, b \in \K \quad f(ax+by) = a f(x) + b f(y)$$
\end{defn}
\spazio
\begin{defn}[Antilineare]\bianco
Sia $V$ un $\C$-spazio vettoriale.
$ f:\, V \to V $ di dice antilineare se 
$$ \forall x,y \in V \quad \forall a, b \in \K \quad f(ax+by) = \overline{a}f(x) + \overline{b} f(y)$$
\end{defn}
\spazio
Grazie a queste 2 definizione possiamo definire i prodotti Hermitiani
\begin{defn}[Prodotto Hermitiano]\bianco
Sia $V$ un $\C$-spazio vettoriale, allora
un prodotto scalare \'e su $V$ una funzione
$$ \phi:\, V \times V \to \C$$ 
tale che
\begin{itemize}
\item[(i)] $\phi$ \'e lineare sulla prima componente
\item[(ii)] $\phi$ \'e antilineare sulla seconda componente
\item[(iii)] $\phi(v,w) = \overline{\phi(w,v)} \quad \forall v,w \in V $
\end{itemize}
\end{defn}
\begin{defn}[Sesquilineare]\bianco
Una funzione con solo le  prime 2 propiet\'a si chiama  forma sesquilineare
\end{defn}
\spazio
\begin{oss} Dalla propiet\'a (iii) segue che $\phi(v,v) \in \R$ quindi possiamo estendere ai prodotti Hermitiani tutte quelle nozioni che abbiamo definito sui prodotti scalari reali (definito positivo, indice di positivit\'a, ...)
\end{oss}
\spazio
Nel caso dei prodotti scalari potevamo definire un prodotto scalare mediante una base $\B$ e valeva che 
$$ \phi(v,w) = [v]^t_\B \cdot A \cdot [w]_\B \quad \text{ con } 
A=M_\B(\phi)=A^t$$
Possiamo reiterare il ragionamento per un prodotto Hermitiano $\psi$ ottenendo
$$ \psi(v,w)= [v]^t_\B \cdot A \cdot \overline{[w]_\B} \quad \text{ con } A = \overline{A^t}$$
\spazio
\begin{defn}[Basi unitarie]\bianco
Sia $\phi$ un prodotto Hermitiano, allora $\B$ \'e un base unitaria se $M_\B (\phi)=I $ 

\end{defn}
\spazio
\begin{defn}[Gruppo unitario]
$$ U(n) = \{ P \in GL(n,\C) \, \vert P^{-1}=\overline{P^t}`$$

\end{defn}
\begin{oss}Le basi unitarie corrispondono alle basi unitarie del prodotto scalare\\
Il gruppo ortogonale \'e il corrispettivo del gruppo ortogonale dei prodotti scalari
\end{oss}
\newpage
\subsection{Teorema spettrale e operatori normali}
Definiamo l'aggiunto in modo analogo al caso dei prodotti scalari
\begin{defn}[Endomorfismi normali]\bianco
$f \in End(V) $ si dice normale se commuta con il suo aggiunto ovvero
$$ f \circ f^\star = f^\star \circ f $$
Alcuni operatori aggiunti sono
\begin{itemize}
\item Autoaggiunto se $f=f^\star$
\item Antiautoaggiunto se $f^\star = -f$
\item Unitario se $f^\star=f^{-1}$
\end{itemize}
\end{defn}
\spazio
\begin{thm}[Teorema spettrale hermitiano]\label{TSpettraleH}\bianco
Sia $V$ un $\C$-spazio vettoriale e sia $\phi$ un prodotto Hermitiano su $V$ definito positivo.\\
$f \in End(V)$
$$ f \text{ unitariamente diagonalizzabile } \quad \ses \quad f \text{ \'e normale } $$
\end{thm}
\spazio
La seguente proposizione d\'a una spiegazione del perch\'e il teorema \ref{TSpettraleH} si chiama teorema spettrale
\begin{prop}Sia $(V, \phi) $ con $V$ uno spazio vettoriale su $\C$ e $\phi $ definito positivo.\\
Sia $f \in End(V)$ normale allora
\begin{enumerate}
\item $f$ \'e autoaggiunto $\ses$ $Sp(f) \subseteq \R$
\item $f$ \'e antiautoaggiunto $\ses$ $Sp(f) \subseteq \C - \R$
\item $f$ \'e unitario $\ses$ $Sp(f) $\'e unitario ovvero formato da autovalori di norma $1$

\end{enumerate}
\proof $$f \text{ normale} \quad \implica \quad \exists \B \text{ unitaria } \tc M_\B(f)=D \text{ diagonale }$$
Inoltre essendo $\B$ unitaria vale che 
$f^\star$ viene rappresentata da $\overline{D}^t=\overline{D}$

\begin{enumerate}
\item  
$$ f = f^\star \quad \ses \quad D=\overline{D} \quad \ses \quad D \text{ reale } \quad \ses \quad Sp(f)\subseteq \R$$
\item 
$$ f=f^{-1} \quad \ses  \quad D=-\overline{D} \quad \ses \quad D \text{ immaginario puro } \quad \ses \quad Sp(f) \text{ immaginario puro } $$
\item 
$$f^{-1}=f^\star \quad \ses \quad D^{-1}= D^\star \quad \ses \quad I = D \overline{D}$$
Ora se $D$ ha sulla diagonale $\mu_1, \cdots , \mu_n $ allora $D \overline{D}$ avr\'a sulla diagonale $\vert \vert \mu_1 \vert \vert ^2, \cdots, \vert \vert \mu_1 \vert \vert ^2$ quindi
$$ I = D \overline{D}\quad \ses \quad \vert \vert \mu_i\vert \vert = 1 \quad \forall i=1,\cdots, n $$
\end{enumerate}
\endproof
\end{prop}
%\end{document}