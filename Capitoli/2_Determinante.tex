%%\documentclass[a4paper,12pt]{article}
\usepackage[a4paper, top=2cm,bottom=2cm,right=2cm,left=2cm]{geometry}

\usepackage{bm,xcolor,mathdots,latexsym,amsfonts,amsthm,amsmath,
					mathrsfs,graphicx,cancel,tikz-cd,hyperref,booktabs,caption,amssymb}
\hypersetup{colorlinks=true,linkcolor=blue}
\usepackage[italian]{babel}
\usepackage[T1]{fontenc}
\usepackage[utf8]{inputenc}


\newcommand{\N}{\mathbb{N}} %naturali 
\newcommand{\R}{\mathbb{R}} %Reali
\newcommand{\K}{\mathbb{K}} % Campo K 
\newcommand{\C}{\mathbb{C}}
\newcommand{\F}{\mathbb{F}}

\newcommand{\B}{\mathfrak{B}} %Base B
\newcommand{\D}{\mathfrak{D}}%Base D
\newcommand{\RR}{\mathfrak{R}}%Base R 
\newcommand{\Can}{\mathfrak{C}}%Base canonica
\newcommand{\Rif}{\mathfrak{R}}%Riferimento affine
\newcommand{\AB}{M_\D ^\B }% matrice applicazione rispetto alla base B e D 

\newcommand{\vett}{\overrightarrow}
\newcommand{\sd}{\sim_{SD}}%relazione sx dx
\newcommand{\sbarra}{\backslash} %% \ 
\newcommand{\ds}{\displaystyle} 
\newcommand{\alla}{^}  
\newcommand{\implica}{\Rightarrow}
\newcommand{\ses}{\Leftrightarrow} %se e solo se
\newcommand{\tc}{\quad \text{ t. c .} \quad } % tale che 
\newcommand{\spazio}{\vspace{0.5 cm}}
\newcommand{\bbianco}{\textcolor{white}{,}}
\newcommand{\bianco}{\textcolor{white}{,} \\}% per andare a capo dopo 																					definizioni teoremi ...
\newcommand{\nvett}{v_1, \, \dots , \, v_n} % v1 ... vn
\newcommand{\ncomb}{a_1 v_1 + \dots + a_n v_n} %a1 v1 + ... +an vn
\newcommand{\nrif}{P_1, \cdots , P_n} 
\newcommand{\bidu}{\left( V^\star \right)^\star}

%per creare teoremi, dimostrazioni ... 
\theoremstyle{plain}
\newtheorem{thm}{Teorema}[section] 
\newtheorem{ese}[thm]{Esempio} 
\newtheorem{ex}[thm]{Esercizio} 

\newtheorem{cor}[thm]{Corollario} 
\newtheorem{lem}[thm]{Lemma} 
\newtheorem{al}[thm]{Algoritmo}
\newtheorem{prop}[thm]{Proposizione} 
\theoremstyle{definition} 
\newtheorem{defn}{Definizione}[section] 
\theoremstyle{remark} 
\newtheorem{oss}{Osservazione} 


% diagrammi commutativi tikzcd
% per leggere la documentazione texdoc

%\begin{document}

\section{Determinante}

\begin{defn}[Determinante]
Sia $n \in \mathbb{N}$\\
Il determinante \`e una funzione:
$$ D: \,M(n, \mathbb{K})\cong \underbrace{ \mathbb{K } \times \dots \mathbb{K}^n}_\text{ n volte}
\to \mathbb{K}$$
che soddisfa queste 3 propriet\'a:
\begin{enumerate}
    \item[(i)] n-lineare rispetto alle colonne
    \item[(ii)] $ D ( \dots,X , X , \cdots )=0$ 
    \item[(iii)] $D(I_n)=1$
\end{enumerate}
\end{defn}
\begin{prop}[Propiet\'a aggiuntive]
Sia $D$ un determinante allora 
\begin{enumerate}
\item $ D (\dots, X,Y, \cdots )=- D (\dots, Y,X, \cdots )$
\item $D (\dots, X , \dots, X \dots )=0 $
\item $D (\dots, X , \dots, Y \cdots )= - D (\dots, Y , \dots, X \dots )$
\end{enumerate}

\proof   \bbianco
\begin{enumerate}
\item  Dalla propiet\'a (ii) segue che 
$$ D ( \dots, X+Y, X+Y, \dots )=0 $$
ora usando la linearit\'a rispetto alle colonne otteniamo 
$$ D(\dots,X,X, \dots )+ D(\dots,X,Y, \dots ) + D(\dots,Y,X, \dots )+ D(\dots,Y,Y, \dots )=0 $$
ovvero usando la propiet\'a (ii)
$$ D (\dots , X,Y,\dots ) = - D ( \dots, Y,X, \dots ) $$
\item Applicando la propiet\'a sopra dimostrata otteniamo che 
$$D (\dots, X , \dots, X \dots )= (-1)^i D ( \dots,X,X,\dots)=0$$
\item Ripercorriamo la prima dimostrazione ed otteniamo la tesi
\end{enumerate}
\endproof
\end{prop}
\newpage
\begin{prop}[Unicit\'a di $D$]\bianco 
Supponiamo che esista una funzione $D$ con le propiet\'a sopra descritte allora tale funzione \`e unica
\proof
$$A= \begin{pmatrix}
a_{11} & \cdots & a_{1n}
\\
\vdots && \vdots\\
a_{n1}& \cdots & a_{nn}
\end{pmatrix}= (a_{11}E^1+ a_{21}E^2+ \dots + a_{n1}E^n, \dots ,
a_{1n}E^1+ a_{2n}E^2+ \dots + a_{nn}E^n) $$
Calcoliamo $D(A)$ e sviluppiamo con la multilinearit\'a:
$$ D(A)=\sum_{\sigma \in S_n} a_{\sigma(1),1}\dots a_{\sigma(n),n} D \left( E^{\sigma(1)} , \dots , E ^{\sigma(n)} \right)$$
Se $D$ esiste allora \`e definita nel seguente modo:
$$ D(A)= \sum_{\sigma \in S_n} (-1)^{P(\sigma)} a_{\sigma(1),1}\dots a_{\sigma(n),n}
$$ 
Dove $P(\sigma)$ indica la parit\'a della permutazione.
\end{prop}
\spazio 
\begin{prop}[Esistenza]\bianco
Esiste una funzione che soddisfa le 3 propiet\'a 
\proof Andrebbe dimostrato che la funzione sopra definita soddisfa veramente le propiet\'a

\end{prop}
\spazio 
\begin{lem}
$$ \Lambda^2 =\{ \phi:\, M(n, \K) \to \K \, \vert \, \phi \text{ soddisfa (i) e (ii)} \}$$
\`e un sottospazio vettoriale e il determinante \`e una base 
\proof La dimostrazione che $\Lambda^2$ \`e un sottospazio \`e lasciata come esercizio .\\
Sia $\phi \in \Lambda^2 $ tale che $\phi (I_n )= \lambda $.\\
Ripercorrendo la dimostrazione dell'unicit\'a otteniamo
$$ \phi(A)=\sum_{\sigma \in S_n} a_{\sigma(1),1}\dots a_{\sigma(n),n} \phi \left( E^{\sigma(1)} , \dots , E ^{\sigma(n)} \right)= \sum_{\sigma \in S_n} (-1)^{P(\sigma)} a_{\sigma(1),1}\dots a_{\sigma(n),n} \phi \left( I_n \right) = \lambda D(A)$$
e poich\`e vale $\forall A \in M(n,\K)$ allora
$$  \forall \phi \in \Lambda^2 \quad \phi = \lambda \cdot D $$
\endproof
\end{lem}
\begin{prop}[Formula di Binet]
$$ D(AB)=D(A)\cdot D(B)$$
\proof Sia $B\in M(n,\K)$.\\
Consideriamo la funzione 
$$ \phi(A) = D (BA) = D(BA^1, \dots, BA^n)$$
Osserviamo che $ \phi \in \Lambda^2 $ infatti
\begin{itemize}
\item Il prodotto di matrici \`e lineare e la composizioni di lineari \`e lineare
\item $\phi(\dots, X ,X, \dots) = D ( \dots, BX, BX, \dots) = 0$
\end{itemize}
Ora per il lemma precedente
$$ \phi (A)= \phi (I_n) \det (A) = D \left( B I_n \right) \cdot D(A) = D(B)\cdot D(A)$$
\endproof
\begin{oss}Essendo $\K$ un campo $ D(B) \cdot D(A)= D(A)\cdot D(B)$ da cui 
$$ D(AB)=D(BA)$$
\end{oss}
\end{prop}
\spazio
\begin{cor}
$$ A \text{ invertibile } \ses D(A) \neq 0 $$
\proof $\implica$ \\
Se $A$ \`e invertibile allora esiste $A^{-1}$
$$1=D(I_n)= D  \left( A A^{-1}\right) = D(A) D\left( A^{-1} \right)$$
dunque $D(A)$ \`e invertibile ovvero \`e diverso da $0$\\ 
$\Leftarrow$ in modo contro nominale.\\
Supponiamo che $A$ non sia invertibile allora esiste un indice $j$ tale che
$$ A^j = \sum_{ i=1  \atop{i\neq j }}^n a_i A^i $$
quindi 
$$D(A)= D \left( A^1 , \dots, A^j , \dots, A^n \right) =a_i D(A^1,\dots, A^1 , \dots, A^n ) + \dots + a_n D(A^1, \dots, A^n , \dots, A^n ) =0$$
\endproof
\end{cor}
\spazio
\begin{prop}\label{Det_trasposta}
$$ D (A) = D(A^t )$$
\proof
$$ D(A)= \sum_{\sigma \in S_n } (-1)^{P(\sigma)} a_{\sigma(1),1}\dots a_{\sigma(n), n } $$
Ora considerando l'involuzione di $S_n$ che manda $\sigma $ in $\sigma^{-1}$ e facendola agire sulla formula otteniamo
$$ D(A)= \sum_{\sigma \in S_n } (-1)^{P(\sigma^{-1})} a_{\sigma^{-1}(1),1}\dots a_{\sigma^{-1}(n), n }=D(A^t)$$
\endproof
\end{prop}
\newpage
\begin{prop}[Calcolo del determinante con Gauss]\bianco
Sia $A \in M (n ,\K) $, applico C-Gauss e ottengo $\hat{A}_C$
\begin{itemize}
\item Se $rk\left( \hat{A}_C \right) =rk(A) < n $ allora $D A =0 $
\item Se $rk\left( \hat{A}_C \right) =rk(A) =n	 $ allora $\hat{A}_C=I_n$  da cui
$$ A E_1 \cdots E_k=I_n $$ 
con $E_1, \cdots , E_k$ matrici C-elementari tali che $ E_1, \cdots , E_k=A^{-1} $ da cui
$$D\left( A^{-1}\right) = \prod_{i=1}^k D( E_i) $$
ma le matrici elementari corrispondono ad azioni di C-Gauss dei tre tipi
\begin{enumerate}
\item Scambio di colonne $D(E)=-1$
\item Moltiplico una colonna per costante $D( E)=c$
\item Somma di una colonna per un multiplo di un altra
$ D(E)= 1$
\end{enumerate}
Da cui $$D\left( A^{-1} \right)= (-1)^\alpha \cdot c_1 \cdot c_s$$
Dove $\alpha$ \`e il numero di operazioni del primo tipo e $c_i$ sono le c-esime costanti per cui moltiplico 
\end{itemize}
\end{prop}

\spazio 
Forniamo una nuova dimostrazione della proposizione~\ref{Det_trasposta}
\begin{cor} $\det A = \det {}^t A $
\proof \bbianco
\begin{itemize}
\item se $A$ non \`e invertibile, ${}^tA$ non lo \`e da cui 
\item Se $A$ \`e invertibile 
$$A=E_1\cdots E_k \quad D ( A)=\prod_j E_j  \implica D(A^t)= \prod_j D(E_j^t) $$
dove l'ultima implicazione \`e lasciata per esercizio e si pu\'o verificare nei 3 casi
\end{itemize}
\endproof
\end{cor}
\newpage

\subsection{Formula di Cramer per sistemi lineari}
Sia $A \in M(n,\K) $.\\
Studiamo il sistema $AX=B$.\\
Sia $X=\begin{pmatrix} x_1 \\ \vdots \\ x_n \end{pmatrix}$ una soluzione allora
$$ B= x_1 A^1 + \cdots + x_n A ^n $$ 
Consideriamo la matrice $$M_j = \left( A^1, \cdots , B , \cdots , A^n \right) $$
dove la j-esima colonna di A e sostituita da B.\\
Il determinante di $M_j$ si pu\'o calcolare usando la multilinearit\'a dunque 
$$ \det M_j =x_1 \cdot \det A $$
Se $\det A \neq 0 $, la soluzione $X$ esiste ed \`e unica 
$$ x_j = \frac{\det M_j }{\det A }$$
\subsection{Calcolo dell'inversa}
Sia $A \in GL(n,\K) $.\\
Troviamo $A^{-1} $ come soluzione $X$ del sistema $AX=I_n$, tale sistema si pu\'o scomporre
$$ \begin{cases}
AX^1=e_1 \\ \vdots \\ AX^n=e_n
\end{cases}$$ risolvibile con Cramer\\
L'inverso di una matrice \`e una speciale funzione razionale delle entrate della matrice di partenza.
\newpage
\subsection{Definizione ricorsiva}
\begin{defn}[Sviluppo di Laplace rispetto ad una riga]\bianco
Diamo una definizione ricorsiva per la formula del determinante $n$-esimo
$$D_1 \left(\begin{pmatrix}
a
\end{pmatrix}\right) =a $$
Fissata una riga $i$ 
$$ D_{n+1}(A)= \sum_{j=1} ^{n+1}  (-1) \alla {i+j} \cdot [A]_{ij}\cdot D_{n} ( A_{ij} ) $$
Dove con la notazione $ A_{ij} \in M(n-1,n-1) $ si indica la matrice ottenuta cancellando la $i$-esima riga e la $j$-esima colonna.
\end{defn}
Per visualizzare i segni possiamo usare alla matrice del segni
$$\begin{pmatrix}
1& -1 & 1 & -1 &\cdots \\
-1& 1 & -1 & 1 &\cdots \\
\vdots &   &   & &  \vdots
\end{pmatrix}$$
\spazio
\begin{prop} Comunque scelgo l'indice di riga, la funzione sopra definita  verifica le tre propiet\'a caratterizzanti del determinante
\proof Procediamo per induzione il passo base \`e dato da $n=2$
\begin{enumerate} 
\item Proviamo la linearit\'a rispetto alla $k$-esima colonna , ovvero dobbiamo provare che $\forall j $ il termine 
$$ (-1)^{i+j} a_{ij}D_{n-1}(A_{ij})$$
non dipende da $k$\\
Se $j=k$\\
Il termine $a_{ij}$ \`e fissato dunque non dipende da $k$
Il termine $D_{n-1}(A_{ij}) $ non dipende da $A^k$ quindi \`e costante.\\
Se $j \neq k $\\
La funzione \`e una composizione di applicazioni lineari infatti si utilizza la proiezione $$ A^k \to A^k_{ij} $$
che \`e lineare
\item $ D( \cdots , X, X , \cdots) =0$ dove le colonne uguali sono $k$ e $k+1$.\\
Sia $j\neq k, k+1 $
Anche in $A_{ij}$ ci sono due colonne adiacente uguali da cui $D_{n-1} (A_{ij})=0$\\
Verifichiamo ora l'altro caso 
$$ a_{ik}D_{n-1}(A_{ik})=a_{i,k+1} D_{n-1}(A_{i,k+1} ) $$ 
ora entrambe le colonne sono uguali quindi \`e vara l'uguaglianza
\item \`e lasciata come esercizio
\end{enumerate}
\endproof
\end{prop}




%\begin{defn} Una matrice si dice diagonalizzabile se \`e simile ad una matrice diagonale \end{defn}

% \end{document}