%\documentclass[a4paper,12pt]{article}
\usepackage[a4paper, top=2cm,bottom=2cm,right=2cm,left=2cm]{geometry}

\usepackage{bm,xcolor,mathdots,latexsym,amsfonts,amsthm,amsmath,
					mathrsfs,graphicx,cancel,tikz-cd,hyperref,booktabs,caption,amssymb}
\hypersetup{colorlinks=true,linkcolor=blue}
\usepackage[italian]{babel}
\usepackage[T1]{fontenc}
\usepackage[utf8]{inputenc}


\newcommand{\N}{\mathbb{N}} %naturali 
\newcommand{\R}{\mathbb{R}} %Reali
\newcommand{\K}{\mathbb{K}} % Campo K 
\newcommand{\C}{\mathbb{C}}
\newcommand{\F}{\mathbb{F}}

\newcommand{\B}{\mathfrak{B}} %Base B
\newcommand{\D}{\mathfrak{D}}%Base D
\newcommand{\RR}{\mathfrak{R}}%Base R 
\newcommand{\Can}{\mathfrak{C}}%Base canonica
\newcommand{\Rif}{\mathfrak{R}}%Riferimento affine
\newcommand{\AB}{M_\D ^\B }% matrice applicazione rispetto alla base B e D 

\newcommand{\vett}{\overrightarrow}
\newcommand{\sd}{\sim_{SD}}%relazione sx dx
\newcommand{\sbarra}{\backslash} %% \ 
\newcommand{\ds}{\displaystyle} 
\newcommand{\alla}{^}  
\newcommand{\implica}{\Rightarrow}
\newcommand{\ses}{\Leftrightarrow} %se e solo se
\newcommand{\tc}{\quad \text{ t. c .} \quad } % tale che 
\newcommand{\spazio}{\vspace{0.5 cm}}
\newcommand{\bbianco}{\textcolor{white}{,}}
\newcommand{\bianco}{\textcolor{white}{,} \\}% per andare a capo dopo 																					definizioni teoremi ...
\newcommand{\nvett}{v_1, \, \dots , \, v_n} % v1 ... vn
\newcommand{\ncomb}{a_1 v_1 + \dots + a_n v_n} %a1 v1 + ... +an vn
\newcommand{\nrif}{P_1, \cdots , P_n} 
\newcommand{\bidu}{\left( V^\star \right)^\star}

%per creare teoremi, dimostrazioni ... 
\theoremstyle{plain}
\newtheorem{thm}{Teorema}[section] 
\newtheorem{ese}[thm]{Esempio} 
\newtheorem{ex}[thm]{Esercizio} 

\newtheorem{cor}[thm]{Corollario} 
\newtheorem{lem}[thm]{Lemma} 
\newtheorem{al}[thm]{Algoritmo}
\newtheorem{prop}[thm]{Proposizione} 
\theoremstyle{definition} 
\newtheorem{defn}{Definizione}[section] 
\theoremstyle{remark} 
\newtheorem{oss}{Osservazione} 


% diagrammi commutativi tikzcd
% per leggere la documentazione texdoc

%\begin{document}

\section{Spazi Vettoriali}

\begin{defn}[Spazio vettoriale]\bianco
Sia $\K$ un campo e sia $V$ un insieme non vuoto sul quale sono definite 2 operazioni:
$$+:\, V \times V  \to V $$
$$\cdot:\, \K \times V \to V $$
che verificano le seguenti proprietà:
\begin{enumerate}
\item $(V,+)$ rende V un gruppo commutativo
\item $\forall \lambda_1, \, \lambda_2 \in \K \, \, \forall v \in V \quad (\lambda_1+\lambda_2)\cdot v = \lambda_1 \cdot v + \lambda_2 \cdot v)$
\item $\forall \lambda \in \K \, \, \forall v_1, \, v_2  \in V \quad \lambda\cdot (v_1+v_2)=\lambda\cdot v_1 + \lambda\cdot v_2$
\item $\forall \lambda_1, \, \lambda_2 \in \K \, \, \forall v \in V \quad (\lambda_1\cdot \lambda_2)\cdot v = \lambda_1 \cdot ( \lambda_2 \cdot v)$
\item $ \forall v \in \K \quad 1 \cdot v = v $
\end{enumerate}
Allora $V$ \`e detto uno spazio vettoriale sul campo $\K$ oppure un $\K$-spazio vettoriale

\end{defn}
\begin{oss}Gli elementi dello spazio vettoriale vengono chiamati vettori mentre gli elementi del campo scalari.\\
Usando la notazione di sopra la prima operazione prende il nome di somma di vettori mentre la seconda prodotto per scalari 
\end{oss}
\begin{ese} $\K \alla n $ \`e uno spazio vettoriale su $\K$.\\
Le 2 operazioni sono cos\`i definite:
$$ \begin{pmatrix}
x_1 \\
\vdots \\
x_n
\end{pmatrix} + \begin{pmatrix}
y_1 \\
\vdots \\
y_n
\end{pmatrix}= 
\begin{pmatrix}
x_1 + y_1 \\
\vdots \\
x_n+ y_n
\end{pmatrix}$$
$$ \lambda \begin{pmatrix}
x_1 \\
\vdots \\
x_n
\end{pmatrix}= \begin{pmatrix}
\lambda \cdot x_1 \\
\vdots \\
 \lambda \cdot  x_n
\end{pmatrix}$$
\end{ese}

\begin{ese} $ \K \alla E = \{ f : \, E \to \K \} $ con le seguenti operazioni 

$$ (f+g) (x)=f(x)+g(x) $$
$$ ( \lambda f ) (x)=\lambda f(x) $$
\`e $\K$-spazio vettoriale.\\
\end{ese}
\begin{ese}
$\K[x] $ con le usuali somme e prodotto per uno scalare di polinomi \`e un $\K$-spazio vettoriale
\end{ese}
\newpage

\subsection{Spazio di matrici}

La matrice è una tabella di numeri organizzati in righe e colonne.\\
Denotiamo con $M(m,n,\K)$  la matrice di taglia $m \times n $ (m righe, n colonne) a coefficienti in $\K$.\\
Ogni elemento della matrice è denotato da 2 indici che indicano rispettivamente l'indice di riga e di colonna.
$$ A=\begin{pmatrix} a_{11} &  \cdots &a_{1n}\\
\vdots & \cdots & \vdots \\
a_{m1} & \cdots & a_{mn}
\end{pmatrix}$$

\begin{defn} Sia $ A \in M(m,n, \K ) $\\
Denotiamo con : 
\begin{itemize}
\item $A_i$ la riga i-eisma
\item $A^j$ la colonna j-esima
\item $[A]_{i,j}$ l'entrata di posto ij della matrice
\end{itemize}
\end{defn}

\spazio

\begin{defn}[Somma tra matrici]\bianco
La somma tra matrici si fa posto per posto.\\
Siano $A, B \in M (m,n,\K ) $ allora $$\forall i= 1,\cdots ,m \quad \forall j=1,\cdots , n\qquad [A+B]_{i,j}=[A]_{i,j}+[B]_{i,j}$$
\end{defn}

\begin{defn}[Prodotto per scalari]\bianco
Il prodotto di una matrice per uno scalare  si fa moltiplicando tutte le entrate della matrice per lo scalare.\\
Siano $A \in M(m,n,\K)$ e $ \lambda \in K $ allora:
$$ \forall i= 1,\cdots ,m \quad \forall j=1,\cdots , n\qquad [\lambda \cdot A]_{i,j}=\lambda [A]_{i,j}$$ 
\end{defn}
\begin{prop}L'insieme $M(m,n,\K)$ con le 2 operazioni sopra definite \`e uno spazio vettoriale sul campo $\K$
\end{prop}
\spazio
\begin{defn}[Prodotto tra matrici] \bianco
Il prodotto tra 2 matrici \`e un operazione cos\`i definita:
$$M(m,k,\K)\times M(k,n,\K)\to M(m,n,\K)\qquad (A,B)\to AB$$ 
$$\forall i,j \quad [AB]_{ij}=\sum_{h=1}^k [A]_{ih}\cdot [B]_{hj}$$
\end{defn}
\begin{oss}Il prodotto tra matrici non \`e commutativo 
\end{oss}
\newpage

\begin{defn}[Diagonale ]\bianco Sia $ A \in M(m,n,\K)$, la diagonale di A è
$$ \{ [A]_{ii} \, | \, 1 \leq i \leq \min(m,n) \} $$
\end{defn}
\begin{defn} Sia $ A \in M(m,n, \K)$ diciamo che A \`e
\begin{itemize}
\item Diagonale se $ \forall i \neq j\quad [A]_{ij}=0$
\item Triangolare superiore se $ \forall i>j \quad [A]_{ij}=0$
\item Triangolare inferiore se $\forall i<j \quad [A]_{ij}=0$
\end{itemize}
\end{defn}

\newpage

\section{Sottospazi vettoriali}
\begin{defn} [Sottospazio vettoriale] Sia V $\K$-spazio vettoriale.\\ 
$ W \subseteq V $ si dice sottospazio vettoriale di V se:
\begin{enumerate}
\item $ 0 \in W $ oppure $ W \neq \emptyset$  
\item $\forall w_1, w_2 \in W \quad w_1+w_2 \in W $ (chiuso per somma)
\item $ \forall \lambda \in \K \, \forall	w \in W \quad \lambda \cdot w \in W $ (chiuso per prodotto per scalari)
\end{enumerate}
Se W è un sottospazio vettoriale di V, W con l'operazione di somma e prodotto ristrette è uno spazio vettoriale
\end{defn}

\begin{prop}\label{intsottospazi}
L' intersezione di 2 sottospazi vettoriali è un sottospazio vettoriale e pi\`u in generale l'intersezione numerabile di una famiglia di sottospazi \`e un sottospazio.
\proof \bbianco
\begin{itemize}
\item Essendo  $A$ e $B$ sottospazi vettoriali
 $$0 \in A \quad  0 \in B$$ 
dunque $$ 0 \in A \cap B$$
\item Siano $x,y \in A\cap B $.\\
Essendo $A$ e $B$ sottospazi vettoriali
 $$  x+y \in A, \,x+y \in B $$
 dunque $$ x+y \in A \cap B$$
 
\item La dimostrazione della chiusura per prodotto di scalari è analoga alla precedente.
\end{itemize}
\endproof
\end{prop}

\begin{oss}
In generale, l'unione di sottospazi vettoriali non è un sottospazio vettoriale.\\
In $\R^2$  due rette distinte passanti per l'origine sono sottospazi di $\R^2$ ma se consideriamo la loro unione essa non \`e chiusa per somma
\end{oss}
\newpage

\begin{defn}[Span X]\bianco
Sia V uno spazio vettoriale e sia $X$ un suo sottoinsieme non vuoto
$$ Span(X)= \bigcap_{ \, X \subseteq W \atop{ W \text{ sottospazio vettoriale di } V}} W $$
\begin{oss}
Span(X) è un sottospazio vettoriale per la proposizione~\ref{intsottospazi}
\end{oss}
\end{defn}
\spazio

\begin{defn}[Combinazione lineare]\bianco
Sia X un sottoinsieme di V (spazio vettoriale).\\
$v \in V $ si esprime come combinazione lineare di elementi di $X$ se 
$$v=\sum _{x \in X} a_x x \quad   a_x \in \K  \quad  \{ x \, \vert \, a_x\neq 0 \} \text{ \`e finito}   $$
\end{defn}

\begin{defn}[Comb X]
$$ Comb(X)= \{ v \in V  \, \vert \, v \text{ si esprime come c.i di }  X \}$$
\end{defn}

\begin{thm} Comb(X) è un sottospazio vettoriale.\\
\proof \bbianco
\begin{itemize}
\item Poich\`e $X \neq \emptyset$  consideriamo la combinazione lineare
 $$\sum_{x \in X } 0 \cdot x  $$ 
 tale combinazione esprime il vettore nullo quindi $ 0 \in Comb (X)$
\item Se $v,w \in Comb(X)$ allora 
$$ v= \sum_{x \in X} a_x x \quad w=\sum_{x \in X } b_x x $$ quindi 
$$
 v+w=\sum_{x \in X} (a_x+b_x)x $$ da cui segue che $ v+w \in Comb(X)$
\item Se $ v \in Comb(X) $ allora 
$$  v= \sum_{x \in X} a_x x $$ quindi 
$\forall \lambda \in \K $ $$ \lambda v = \lambda \sum_{x \in X }a_x x = \sum_{x \in X} (\lambda a_x) x $$ ovvero $\lambda v \in Comb(X)$
\end{itemize}

\endproof
\end{thm}

\begin{lem}[Minimalit\'a di Comb X]\label{minimalita} \bianco
Sia $V$ uno spazio vettoriale \\
Sia $S$ un sottospazio vettoriale di $V$   e $X$ un sottoinsieme finito non vuoto di $V$
$$X\subset S \subseteq Comb(X) \quad \implica  \quad S =Comb(X)$$
\proof  \bianco$ \subseteq $ per tesi  \\
$\supseteq$ Sia $X=\{ x_1, \, \cdots , \, x_n \}$
$$\forall x \in Comb(X)  \quad  x=\lambda_1 x_1 + \cdots + \lambda_n x_n  \quad \text{ dove   } \lambda_i  \in \K  \quad \forall i =1, \dots \, n $$
 ora da $ X \subset S $ segue che $x_i \in S $ e poich\`e $S$ \`e un sottospazio vettoriale $\lambda_1 x_1 + \cdots + \lambda_n x_n \in S $
\endproof
\begin{oss}Il lemma precedente ci dice che $Comb(X)$ \`e il pi\`u piccolo (rispetto all'inclusione) sottospazio vettoriale che contiene l'insieme $X$
\end{oss}
\end{lem}
\spazio

\begin{thm} Comb(X)=Span(X)
\proof  \bbianco
\begin{enumerate}
\item $ X\subseteq Span (X) $\\
 infatti tutti i $W $ che interseco per ottenere $Span(X) $ contengono $X$, quindi anche la loro intersezione lo contiene
\item $Span(X)\subseteq Comb (X) $ \\
infatti  $X\subseteq Comb(X) $ in modo ovvio, dunque poich\`e l'intersezione \`e pi\'u piccola (rispetto l'inclusione) abbiamo la disuguaglianza
\end{enumerate}
Dai 2 punti ottengo $$ X \subseteq Span(X) \subseteq Comb(X) $$ 
dunque per la minimalit\'a di $Comb(X)$(Lemma~\ref{minimalita}) segue la tesi.
\end{thm}
\newpage
\subsubsection{Somma di sottospazi}
\begin{defn}[Somma di sottospazi]\bianco
Siano $W_1 \text{ e } W_2 $ sottospazi vettoriali di un medesimo spazio.\\
Allora 
$$ W_1+W_2 = Span (W_1 \cup W_2)$$ 
\begin{oss}
Possiamo giustificare la notazione infatti 
$$ \forall v \in W_1 + W_2 \quad \exists w_1 \in W_1 \, w_2 \in W_2 \qquad v=w_1+w_2 $$
\end{oss}
\end{defn}
\spazio 
\begin{defn} [Somma diretta] \bianco
 Se $W_1 \cap W_2=\{0\}$ allora denotiamo 
 $ W_1+W_2$ con 
$$W_1  \oplus W_2 $$ 
tale somma prende il nome di somma diretta
\end{defn}


\begin{thm} Se la somma tra $W_1$ e $W_2$ è diretta allora $$\forall z \in W_1 +  W_2 \quad  \exists ! w_1 \in W_1 \, w_2 \in W_2 \quad z=w_1+w_2$$
\proof Supponiamo per assurdo che 
$$z=w_1+w_2=w'_1+w'_2 \text {  con }w_1, w'_1 \in W_1 \quad w_2,w'_2 \in W_2$$ allora 
$w_1-w'_1=w_2-w'_2$ quindi poiché $W_1$ e $W_2$ sono chiusi rispetto alla somma $$w_1-w'_1 \in W_1 \quad w_2-w'_2 \in W_2$$ da ciò segue che $$ w_1-w'_1=w_2-w'_2 \in W_1 \cap W_2=\{ 0 \} \quad \implica \quad w_1= w'_1\quad w_2=w'_2$$ \endproof
\end{thm}

\newpage

\section{Applicazioni lineari}
\begin{defn}[Applicazione lineare]\bianco
Siano V e W spazi vettoriali su uno stesso campo $\K$.\\
Una funzione $f:\, V \to W $ si dice lineare se 
\begin{enumerate}
\item $ \forall v,v' \in V \quad f(v+v')=f(v)+f(v')$ (rispetta la somma)
\item $\forall \lambda \in \K , \, \forall v \in V \quad f(\lambda \cdot v )= \lambda \cdot f(v)$ (rispetta il prodotto per scalari)
\end{enumerate}
\end{defn}
Diamo una definizione equivalente
\begin{defn}
Sia $f$ come sopra.\\
$f$  \`e lineare se trasforma combinazioni lineari di $V$ in combinazioni lineari di $W$ con gli stessi coefficienti

\end{defn}
\begin{oss} Se f è lineare allora $f(0)=0$.\\
$$f(0)=f(0+0)=f(0)+f(0) \Rightarrow f(0)=0$$
\end{oss}
\vspace{0.5 cm}

\begin{defn} [Omomorfismi] \bianco
Siano V e W spazi vettoriali sullo stesso campo $\K$.
$$Hom(V,W)=\{ f: V \to W \, | \,\text{  lineare } \}$$ 
\end{defn}
\begin{prop} $Hom(V,W)$ è un sottospazio vettoriale.
\end{prop}
\spazio

\begin{defn}[Endomorfismi] \bianco
Sia $V$ uno spazio vettoriale.\\
Sia  $f:\, V \to V $ un'applicazione lineare allora $f$ prende il nome di endomorfismo di $V$.\\
L'insieme degli endomorfismi di uno spazio vettoriale si indica con
$$ End(V)=Hom(V,V) $$
\end{defn}
\spazio

\begin{defn}[Isomorfismi]\bianco
Sia $f\in Hom(V,W)$, $f$ \`e un isomorfismo se 
\begin{itemize}

\item $f$ \`e bigettiva
\item $f^{-1}$ \`e lineare ovvero $ f^{-1} \in Hom(W,V)$

\end{itemize}
\end{defn}
\begin{thm} Sia $f \in Hom(V,W)$
$$ f \text{ bigettiva } \quad \implica \quad f \text{ \`e un isomorfismo }$$
\proof Chiamiamo $g=f^{-1}$\\
Essendo $f$ bigettiva
$$\forall w_1, w_2 \in W \quad  \exists ! v_1, v_2 \in V \tc  f(v_1)=w_1 \text { e }  f(v_2)=w_2$$
Dalla linearit\'a di $f$ otteniamo 
$$ f ( \lambda_1 v_1 + \lambda_2 v_2 ) = \lambda_1 w_1 + \lambda_2 v_2  \quad \forall \lambda_1, \lambda_2 \in \K $$
e applicando ad entrambi i membri $g$
$$ \lambda_1 g(w_1)  + \lambda_2 g(w_2)= g( \lambda_1 w_1 + \lambda_2 w_2) \quad \forall w_1, w_2 \in W \quad \forall \lambda_1, \lambda_2 \in \K$$
\endproof
\end{thm}
\newpage

\begin{thm}[Composizione] Siano V, W, Z spazi vettoriali  su uno stesso campo 
$$ f \in Hom(V,W) \, g \in Hom (W,Z) \quad \implica \quad f\circ g \in Hom (V,Z)$$
\proof Dobbiamo provare che $g\circ f $ \`e lineare.\\
Mostriamo solamente che la composizione rispetta la somma (la verifica per il prodotto \`e analoga)
$$ \forall v_1, v_2 \in V \quad (g \circ f)(v_1 + v_2)=g(f(v_1+v_2))$$
sfruttando la linearit\'a di $f$ 
$$ (g \circ f)(v_1 + v_2)= g(f(v_1)+f(v_2))$$
ora, anche $g$ \`e lineare quindi:
$$ (g \circ f)(v_1 + v_2)= (g\circ f)(v_1) + (g\circ f)(v_2)$$
\endproof
\end{thm}

\vspace{0.5 cm}

\begin{defn}[Gruppo lineare di V]\bianco
$$ GL(V)=\{ f \in End (V) \, | \, f \, invertibile \}$$
\begin{thm}[($(GL(V), \circ )$ è un gruppo ]\bianco
\proof \bbianco
\begin{itemize}
\item Se $f,g \in End(V)$ anche $ f\circ g \in End(V)$.\\
$f\circ g: \, V \to V $ e la composizioni di funzioni lineari è lineare
\item $ 	\circ $ è associativa $g((f+h))(v)=g(f(v)+h(v))= g\circ f(v)+g \circ h(v)$
\item $id_V $ è elemento neutro infatti $ id_V \circ f = f \circ id_V=f$
\end{itemize}
\endproof
\end{thm}
\begin{oss} In generale il gruppo lineare di V non è abeliano (non vale la proprietà commutativa)
\end{oss}
\end{defn}
\newpage

\subsection{Alcuni sottospazi indotti da $f$}
\begin{prop}[Immagine]\bianco
L'immagine della funzione lineare $f: \, V \to W $ è un sottospazio vettoriale.
\proof \bbianco
\begin{enumerate}
\item  $f(0)=0$ quindi $0 \in f(V) $
\item $w_1, w_2 \in f(V)\quad \implica \quad \exists v_1, v_2 \in V \quad f(v_1)=w_1 \, \,  f(v_2)=w_2$
$$w_1+w_2=f(v_1)+f(v_2)=f(v_1+v_2) \in f(V)$$
\item La dimostrazione del fatto che sia chiuso rispetto al prodotto è analoga
\end{enumerate}
\end{prop}
\spazio

\begin{defn}[Nucleo] \bianco
Sia $ f\in Hom(V,W) $
$$Ker (f)= f^{-1}(\{ 0 \})=\{  v \in V \, | \,  f(v)=0 \}$$
\end{defn}

\begin{prop}
$ Ker(f)$ è un sottospazio vettoriale di $V$
\proof \bbianco 
\begin{enumerate}
\item $ 0 \in Ker(f)$ infatti se f è lineare $f(0)=0$
\item Se  $v_1, v_2 \in Ker(f) $ allora $ f(v_1)=f(v_2)=0 $ dunque  $$f(v_1+v_2)=f(v_1)+f(v_2)=0$$ La somma di 2 elementi del nucleo appartiene al nucleo.
\item Se $v_1 \in Ker (f)$ allora $f(v_1)=0 $ dunque 
$$f(\lambda  v ) = \lambda f(v)=\lambda \cdot 0 = 0 $$ Il prodotto di un elemento di un nucleo per un qualsiasi scalare appartiene al nucleo.
\end{enumerate}
\endproof
\end{prop}


\begin{prop} Se f è una funzione lineare, $$ \text{f è iniettiva }\Leftrightarrow \, Ker (f)=\{ 0 \}$$
\proof $\Rightarrow $ \\ 
Poich\`e $f$ \`e lineare allora $ f(0)= 0 $.\\
Ora poich\`e $f$ \`e   iniettiva 
$$\forall v \in V, \, v \neq 0 \quad f(v)\neq f(0)= 0$$ 
dunque $Ker(f) = \{ 0 \}$ 
$\Leftarrow $ Siano $v_1, v_2 \in V $ tale che $f(v_1)=f(v_2)$ allora 
$$ f(v_1) - f(v_2)=f(v_1-v_2)=0 \quad \implica \quad v_1-v_2 \in Ker (f)$$
Ora poich\`e il nucleo \`e ridotto al solo $\{0\}$ $v_1-v_2=0 $ ovvero $v_1=v_2$
\endproof
\end{prop}

\subsection{Matrici e applicazioni lineari}\label{Matrici_Come_Lineari}
\begin{prop}
Ogni matrice induce un'applicazione lineare.
\proof
Sia $ A \in M(m,n,\K ) $ allora definiamo 
$$ L_A:\, \K \alla n \to \K \alla  m \quad X \to A \cdot X $$ 
l'applicazione appena definita \`e lineare poich\`e lo \`e la moltiplicazione di matrici 
\endproof
\end{prop}

\begin{prop} Ogni applicazione lineare \`e indotta da una matrice. 
$$ \forall g \in Hom ( \K\alla m , \, \K \alla n  ) \quad \exists ! A \in M( n,\, m ,\, \K ) \quad t.c \quad g=L_A$$
\proof
Da 
$$ g(e_1) = A \begin{pmatrix}
1 \\ 0 \\ \vdots \\ 0 
\end{pmatrix} = A^1 $$ 
deduco che l 'unica matrice possibile \`e  della forma 
 $$ A = ( g( e_1 ) , \, \cdots , \, g ( e_m )) $$
 Verifichiamo che con questa scelta di A, si verifica che $$ g(X)= A \cdot X  \quad \forall X \in \K \alla m $$
 $$ A \cdot X = x_1 g( e_1 ) + \cdots + x_m g( e_m ) = g ( x_1 e_1 + \cdots + x_m e_m ) = g \begin{pmatrix}
 x_1 \\ \vdots \\ x_m 
 \end{pmatrix}= g(X)
 $$ 
 \endproof
\end{prop}
Mettendo insieme le proposizioni precedenti otteniamo
\begin{prop} $$ M ( m ,\, n ,\, \K ) \cong Hom ( \K \alla n , \, \K \alla m ) $$ 
\end{prop}
Data questa "uguaglianza" tra applicazioni lineari e matrici, a volte, useremo la notazione "la funzione $A$" sottintendendo la funzione lineare associata ad $A$ $(L_A)$
\newpage

\subsection{Alcune applicazioni sulle matrici}
\begin{defn}[Trasposta] $$ {}^t:\, M(m,n,\K) \to M(n,m,\K) \quad A \to A^t$$
 $$\text{con } [A^t]_{ij}=[A]_{ji}$$
\end{defn}
\begin{prop}La trasposta è lineare.
\end{prop} 
\begin{defn}Sia $A \in M(m,n,\K)$ A si dice :
\begin{itemize}
\item simmetrico se $A^t = A$
\item antisimmetrico se $A^t=-A$
\end{itemize}
Sia inoltre:
\begin{itemize}
\item $\mathcal{S}_n=\{ A \in M(n,\K)\, | \, A^t=A\}$ 
\item $\mathcal{A}_n=\{ A \in M(n,\K)\, | \, A^t=-A\}$
\end{itemize}
\end{defn}
\begin{prop} se in $ \K $ $ 2 \neq 0 $ allora 
$$ M(n,\K)= \mathcal{S}_n \oplus \mathcal{A}_n$$

\proof \bbianco
\begin{itemize}
	\item In modo ovvio vale che $S_n + A_n \subseteq M(n,\K)$.\\
	Andiamo a mostrare l'altra inclusione 
	$$ \forall B \in M(n,\K) \quad B+B^t \in \mathcal{S}_n \quad B-B^t \in \mathcal{A}_n $$
	inoltre
	$$ B = (B+B^t) + (B-B^t)$$
	\item Sia $B \in \mathcal{S}_n \cap \mathcal{A}_n$ allora
	$$ B^t=B \quad B\text{ \`e simmetrica}$$
	$$ B^t =-B \quad B \text{ \`e antisimmetrica}$$
	dunque 
	$$2A=0\quad A=0 $$
\end{itemize}
\endproof
\end{prop}
\spazio
\begin{defn}[Traccia]
$$ tr: \, M (n,\K) \to \K \quad tr( A) = \sum_{i=1}\alla n [A]_{ii}$$
Ovvero una funzione che associa ad ogni matrice la somma degli elementi della diagonale
\end{defn}
\begin{prop} La traccia \`e un'applicazione lineare
\end{prop}
\newpage
\section{Basi e dimensioni}
\begin{defn}[Finitamente generato]\bianco
Sia $V$ un $\K$-spazio vettoriale.\\
V  è finitamente generato se $$ \exists \nvett \in V  \tc  V= Span( \nvett ) $$
In tal caso $ \{\nvett \} $ \`e detto insieme di generatori di $V$ 
\end{defn}
\spazio 
\begin{defn}[Indipendenza lineare]  \bianco
Siano $\nvett  \in V$ ($\K$-spazio vettoriale), essi sono linearmente indipendenti  se
$$ \forall \sum_{i=1}^n a_i v_i = 0 \quad \implica \quad a_i = 0 \quad \forall i=1, \dots ,n $$
ovvero se $0$ si esprime come combinazione lineare di $\{ \nvett \} $ allora tutti i coefficienti della combinazione devono essere nulli. 
\end{defn}

\begin{prop} $ \nvett$ sono dipendenti $\Leftrightarrow \exists i \in [ 1, \, n ] \quad v_i \in Span( v_1, \, \dots \, \cancel{v_i} \, \dots ,\, v_n ) $
\proof $ \Rightarrow $ \\
Se $\nvett $ sono dipendenti allora $\ncomb =0 $ e $ \exists i \quad a_i\neq 0 $
$$ v_i=-a_i \alla { -1 } ( a_1 v_1 + \dots +  \cancel{ a_i v_i } + \dots + a_n v_n) $$
\proof $\Leftarrow $ \\
Supponiamo che quello $v_1 $ appartenga allo Span degli altri.
$$ v_1= a_2 v_2 + \cdots + a_n v_n $$
quindi se considero questa combinazione 
$$a_1 v_1 - ( a_2 v_2 +\cdots + a_n v_n )= 0 $$
ma i coefficienti possono anche non essere tutti 0 $ a_1=1 $ e gli altri 1 
\endproof
\end{prop}

\spazio
\begin{defn}[Base]\bianco 
Un insieme ordinato $ \{ \nvett \} $  di vettori di V \`e una base di V  se 
\begin{itemize}
\item $ \nvett $ sono linearmente indipendenti 
\item  $\{ \nvett \} $ \`e un insieme di generatori
\end{itemize}
\end{defn}
\newpage
\begin{prop}[Algoritmo di estrazione ad una base] \bianco
Da ogni insieme finito di generatori non nulli si pu\'o estrarre una base .\\ \\ 
Sia $ X= \{ \nvett \} $ un insieme di generatori non nulli.\\
L'algoritmo \`e definito in modo induttivo, ad ogni passo dell'algoritmo si avr\'a una situazione del tipo $$ Y \, \vert \, X $$
\begin{itemize}
\item Passo 1 
$$ \begin{cases} Y_1 = \{ v_1 \} \\ X_1 =\{ v_2,\, \dots, \, v_n \} 
\end{cases} $$
\item Regola di passaggio.\\
Al passo m-esimo sia $X_m =\{ x_0 , \dots \} $
\begin{itemize}
\item se $ x_0 \in Span (Y_m ) $  $$ \begin{cases} Y_{m+1}=Y_m  \\ X_{m+1} = X_m - \{ x_0 \}   \end{cases}$$
\item se $ x_0 \not \in Span (Y_m ) $  $$ \begin{cases} Y_{m+1}=Y_m \cup \{ x_0 \}   \\ X_{m+1} = X_m - \{ x_0 \}   \end{cases}$$
 
 \end{itemize}

\end{itemize}
L'algoritmo termina quando si realizza la configurazione $ Y \, \vert , \, \emptyset$ e $Y$ \`e la base voluta 
\end{prop}
\begin{lem} L'algoritmo che trasforma X in Y genera una base di V .\\
Occorre dimostrare :
 \begin{enumerate}
\item L'algoritmo termina
\item$Y$ è linearmente indipendente
\item $Y$ genera $V$
\end{enumerate}

\proof \bianco Sia $ Y=\{ y_1,\cdots , y_m \}$ dove $y_i = v_j  \quad \forall i =1 , \, \dots , \, m $
\begin{enumerate}
\item L'algoritmo termina in un numero finito di passaggi infatti ad ogni passo $$\vert X_{m+1} \vert = \vert X_m \vert - 1 $$
\item Per assurdo suppongo che $Y$ non è formato da vettori linearmente indipendenti allora  $$\exists \sum_{j=1}^n a_j y_j =0 \quad \text{ con } \quad a_j \neq 0$$
Sia $k = \max \{ i \, \vert \,  a_i \neq 0 \} $ ($k$ esiste perch\`e l'insieme non \`e vuoto)
 $$
\sum_{j=1}	^k a_j y_j =0 \quad \implica \quad 
a_k y_k=-\sum_{j=1}^{k-1} a_j v_j \quad \implica \quad 
y_k=\sum_{j=1}^{k-1} b_j y_j
$$
quindi $y_k$ è combinazione lineare dei vettori che lo precedono ma questo è assurdo  per come funziona l'algoritmo 
\item 
$Y$ genera $V$ poiché $X$ genera $V$  ed  vettori di $X$ che vengono esclusi da $Y$ si possono ottenere come combinazione lineare di quelli che restano in $Y$
\end{enumerate}

\endproof
\end{lem}

\begin{prop}
[Algoritmo di estensione ad una base]\bianco
Se  lo spazio \`e finitamente generato, da ogni insieme finito di vettori linearmente indipendenti si si pu\'o estrarre una base 
\proof  Sia
$$X=\{ x_1,\dots , x_n \} \text{ un  insieme di vettori linearmente indipendenti} $$
Sia $V$ uno spazio vettoriale finitamente generato, quindi
 $$ \exists Z=\{ z_1,\dots ,z_k \} \text{ un insieme di generatori } $$
Sia $$\tilde{ X} = \{ x_1 , \dots , x_n, z_1, \dots , z_k \} $$
tale insieme genera poich\`e contiene $Z$, dunque possiamo applicare l'algoritmo di estrazione ottenendo una base $Y$.\\
Inoltre per come opera l'algoritmo $ X \subseteq Y$ dunque ho esteso $X$ ad una base di $V$  
\endproof
\end{prop}
\newpage
\begin{lem} [Valori su una base] \label{Valori_Base}\bianco
Un'applicazione lineare $f \in Hom( V, W) $  \`e ben definita se si assegnano i valori di $f$ solamente sui  vettori di una base di $V$.
\proof
Sia $\B= \{ \nvett \} $ una base di $V$.\\
Supponiamo di aver assegnato
$$ f(v_1) = w _1 $$ $$\vdots $$ $$ f(v_n)= w_n  $$ $$ \text{ con } w_i \in W  $$
Proviamo che $\forall v \in V $ \`e ben definito il valore di $f(v)$.\\
Dal fatto che $\B$ \`e una base di $V$ ne segue che 
$$\exists a_1, \dots , a_n \in \K \quad  v= \ncomb $$ 
Dunque sfruttando la linearit\'a di $f$ otteniamo 
$$f(v)= a_1 f(v_1) + \cdots + a_n f(v_n) = a_1 w_1 + \cdots + a_n w_n$$
\endproof
\begin{oss} Il lemma precedente dimostra molto di pi\'u infatti dice che esiste un' unica applicazione lineare che manda una base in vettori preassegnati 
\end{oss}
\end{lem}

\newpage
\subsection{Dimensioni}
\begin{prop} Sia $V$ uno spazio vettoriale finitamente generato e siano 
 \begin{itemize}
\item $X$ un insieme di generatori 
\item $Z$ è linearmente indipendente
\end{itemize}
allora $$\vert X \vert \geq \vert Z \vert $$
\proof Siano $$ X=\{ x_1, \dots , x_n \} $$ 
$$ Z=\{ z_1, \dots , z_k \}$$
Se considero l'insieme ordinato  $ {z_1} \cup X $ genera poiché $X$ genera; tale insieme non non è linearmente indipendente infatti $z_1 \in Span (X)$ ($X$ genera).\\
Comincio ad applicare l'algoritmo di estrazione (comincio da $z_1$) finché non elimino il primo elemento (esiste poiché per quanto detto sopra l'insieme non è linearmente indipendente) .\\ Aggiungo anche $z_2$ all'insieme ottenuto con il primo algoritmo e riapplico l'algoritmo (un altro $x_j$ viene eliminato per lo stesso motivo di prima).\\
Iterando si possono verificare 2 diverse possibilità:
\begin{enumerate}
\item Introduco tutti gli $z_j$ quindi $\vert Z \vert  \leq \vert  X \vert $
\item Se $n < k $ riesco ad introdurre solamente $z_1, \dots, z_n $.\\
 dunque $\{ z_1, \dots, z_n \} $ genera  $V$  quindi $z_{n+1} \in Span(z_1,\dots , z_n) $ ma ciò è assurdo poiché l'insieme $Z$ è linearmente indipendente.
\end{enumerate}
\endproof
\end{prop}
\begin{cor} Se $X$ e $Y$ sono basi di $V$ (spazio vettoriale finitamente generato) allora 
$$ \vert X  \vert = \vert Y \vert$$
\proof\bbianco
\begin{itemize}
\item $ \geq $ 
\\
$X$ \`e un insieme di generatori \\
$Z$ è formato da vettori linearmente indipendenti $$\vert X  \vert \geq \vert Z \vert$$
\item $\leq $\\
$Z$ \`e un insieme di generatori \\
$X$ è formato da vettori linearmente indipendenti
 $$\vert X\vert \leq \vert Z \vert $$
\end{itemize}
Poiché valgono entrambe le disuguaglianze $\vert X  \vert= \vert Z \vert $
\endproof
\end{cor}
Grazie al  corollario precedente è possibile introdurre la seguente definizione:
\begin{defn}[Dimensione] \bianco Sia V uno spazio vettoriale allora definiamo la dimensione di V
 $$ \dim V= \vert X \vert $$ 
 dove X è una base arbitraria di V.
\end{defn}
\newpage

\begin{prop}[Formula delle dimensioni]\bianco \label{dim_ker_imm}
Sia $f \in Hom(V,W) $ allora 
$$ \dim V=\dim Im f +\dim Ker f $$

\proof Sia 
$$ \{ z_1, \dots , z_s \} \text{ una base del nucleo} $$
estendiamola  a 
$$ \{ z_1, \dots, z_s, x_1, \dots, x_n \}\text{ base di } V $$
Mostriamo che 
$$ \{ f(v_1), \dots,f(v_n)\} $$
\`e una base dell'immagine.
\begin{itemize}
\item L'insieme genera
$$ \forall w\in Im (f) \quad \exists v \in V \quad w=f(v)$$ 
Dunque 
$$w= f \left( \sum_{i=1}^s  a_i z_i + \sum_{i=1}^n b_i v_i \right)=\sum_{i=1}^s a_i f(z_i) + \sum_{i=1}^n b_i f(v_i) = \sum_{i=1}^n b_i f(v_i) $$
\item L'insieme \`e formato da vettori linearmente indipendenti.\\
Sia 
$$  \sum_{ i=1}^n a_i f(v_i) = 0 $$
dunque per linearit\'a
$$ f\left(  \sum_{i=1}^n a_i v_i \right) =0 $$ ovvero 
$\ds   \sum_{i=1}^n a_i v_i  \in Ker f$ da cui
 $$ \sum_{i=1}^n a_i v_i  = \sum_{i=1}^s b_i z_i $$
 $$  \sum_{i=1}^n a_i v_i  - \sum_{i=1}^s b_i z_i =0$$
L'ultima combinazione esprime il vettore nullo come combinazione lineare di elementi di una base quindi, in particolare, $ a_i= 0 \quad \forall i=1, \dots, n $\\
Ovvero
$$  \sum_{ i=1}^n a_i f(v_i) = 0 \quad \implica \quad  a_i= 0 \quad \forall i=1, \dots, n $$

\end{itemize}
\endproof
\end{prop}
\begin{oss} La proposizione appena enunciata oltre a dimostrare la formula delle dimensioni ci fornisce anche un modo per poter costruire una base dell'immagine di f.
\end{oss}
\spazio
\begin{cor}\label{iso_base_in_base}$f \in Hom(V,W)$
$$ f\text{ isomorfismo } \ses f \text{ manda una base di } V \text{ in una base di } W $$
\proof $\implica$ Sia $\B =\{ \nvett \} $ una base di $V$.\\
Nella dimostrazione della formula delle dimensione abbiamo dimostrato che $f(\B)$ \`e una base dell'immagine di $f$, ora dal fatto che $f$ \`e un isomorfismo segue che $Im f = W $ da cui $f(\B)$ \`e una base di $W$\\
$\Leftarrow$ 
Sia $\B$ come sopra e sia $w_i = f(v_i ) \quad \forall i=1, \dots, n $.\\
Supponiamo che $\{ w_1, \dots, w_n \} $ \`e una base di $W$.\\
L'applicazione $f$ risulta dunque invertibile infatti $f$ ammette un' inversa $g$ 
$$ g:\, W \to V \quad g(w_i)=v_i \quad \forall i =1, \dots, n $$ 
Osserviamo che $g$ \`e ben definita perch\`e \`e costruita assegnando i valori su una base 
\endproof
\end{cor}
\spazio
\begin{prop}[Invariante completo isomorfismo]\bianco
$$ \text{V,W sono isomorfi } \Leftrightarrow \dim V=\dim W$$
\proof $\implica$\\
Se $V$ e $W$ sono isomorfi allora esiste $f:\, V \to W$ isomorfismo e per il corollario~\ref{iso_base_in_base} $f$ manda una base di $V$ in una di $W$ da cui l'uguaglianza delle dimensioni
\proof $ \Leftarrow $\\
Sia $$\{ \nvett \} \text{  base di } V$$ e
$$ \{ w_1, \dots , w_n \} \text{ base di } W $$
Sia $$f:\, V \to W \qquad \sum_{i=1}^n a_i v_i \to \sum_{i=1}^n a_i w_i  $$
\begin{itemize}
\item $f$ \`e lineare 
$$\forall v,w \in V\quad  v= \sum_{i=1} a_i v_i \quad w=\sum_{i=1}^n b_i v_i $$
$$ f(v+w) = f \left( \sum_{i=1}^n (a_i + b_i) v_i \right)= \sum_{i=1}^n (a_i+ b_i ) w_i = f(v) + f(w) $$
\item $f$ \`e biettiva infatti si pu\'o costruire la funzione inversa
$$ f^{-1}: \, W \to V \quad  \sum_{i=1}^n a_i w_i \to \sum_{i=1}^n a_i v_i $$
\end{itemize}
\endproof
\begin{oss} Dalla proposizione segue che ogni spazio vettoriale è isomorfo ad uno spazio standard della stessa dimensione di V\\  se $\dim V=n $ allora $$ V \cong \K^ n $$.\\
Infatti $\dim \K^ n  =  n $  poich\`e $e_1 , \, \cdots, \, e_n $ \`e una sua base
\end{oss}
\end{prop}
\newpage
\begin{prop}[Formula di Grassman]\bianco
Siano $W$ e $ Z$ sottospazi di $V$ finitamente generato.
$$ \dim (W+Z) = \dim W + \dim Z - \dim(W\cap Z ) $$
Sia 
$$ \D = \{ t_1, \dots , t_s \} \text{ base di } V \cap Z $$ 
estendiamola a 
$$ \B_W =\{ t_1, \dots, t_s, w_1, \dots, w_k \} \text{ base di } W $$
e 
$$ \B_Z=\{t_1, \dots , t_s, z_1, \dots, z_n \} \text{ base di } Z $$
Mostriamo che $\B=\B_W \cup \B_Z$ \`e base di $W+Z$
\begin{itemize}
\item $\B$ \`e un insieme di generatori.
$$ \forall v \in W + Z \quad \exists w \in W \, z \in Z \quad v = w+z $$
Ora 
$$ w = \sum_{i=1}^s a_i t_i + \sum_{i=1}^k b_i w_i $$
$$ z = \sum_{i=1}^s c_i t_1+ \sum_{i=1}^n d_i z_i$$
dunque 
$$ \forall v \in W+Z \quad v= w+z =\sum_{i=1}^s (a_i + c_i) t_1 + \sum_{i=1}^k b_i w_i + \sum_{i=1}^n d_i w_i$$
\item L'insieme $\B$ \`e formato da vettori linearmente indipendenti.\\
Supponiamo che 
$$ \sum_{i=1}^s a_i t_i + \sum_{i=1}^k b_i w_i +\sum_{i=1}^n c_i z_i=0$$
allora
$$ \sum_{i=1}^s a_i t_i + \sum_{i=1}^k b_i w_i =- \sum_{i=1}^n c_i z_i $$
dunque

$$ \sum_{i=1}^n c_i z_i \in W \cap Z \quad \implica \quad  \sum_{i=1}^n c_1 z_1 = \sum_{i=1}^s d_i t_i \quad \implica \quad \sum_{i=1}^n c_i z_i - \sum_{i=1}^s d_i t_i =0 \quad \implica$$
$$ \implica \quad  c_i = 0 \quad \forall i = 1, \dots, n \quad d_j =0 \quad \forall j=1, \dots,s $$
inoltre 
$$ \sum_{i=1}^s a_i t_i + \sum_{i=1}^k b_i w_i \in W\cap Z  \quad \implica \quad  \sum_{i=1}^s a_i t_i + \sum_{i=1}^k b_i w_i = -\sum_{i=1}^s d_i t_i  \quad \implica \quad $$
$$ \sum_{i=1}^s (a_i + d_i)t_i + \sum_{i=1}^k b_i w_i=0 \quad \implica \quad a_i=0 \quad \forall i=i, \dots, s \quad b_j =0 \quad \forall j=1,\dots, k $$
\end{itemize}
\endproof

\end{prop}
\newpage

\subsection{Coordinate}
\begin{prop} [Unicità della combinazione] Sia $\B$ una base di V con n elementi allora $$ \forall v \in V \quad v = \sum_{j=1}^n a_j v_j \text{ è unica }$$
\proof Supponiamo che 
$$ v= \sum_{i=1}^n b_i v_i $$
Allora 
$$ \sum_{i=1}^n a_i v_i = \sum_{i=1}^n b_i v_1 \quad \implica\quad \sum _{i=1}^n (a_i - b_i) v_i = 0 \quad \implica \quad a_i - b_i = 0 \quad \implica \quad a_i = b_i  \quad \forall i = 1, \dots, n $$
\endproof
\end{prop}

Grazie alla precedente proposizione \`e possibile definire le coordinate
\begin{defn}[Coordinate] \bianco 
Le coordinate del vettore $v$ rispetto alla base $\B$ sono i coefficienti dell'unica combinazione lineare che esprime $v$.\\
Tale coordinate si indica con $\ds [v]_\B $
\end{defn}
\spazio
\begin{prop} $$[\,]_\B : \, V \to \K^n \quad v \to [v]_\B$$
è un isomorfismo di spazi vettoriali.
\proof Fissiamo $\B=\{ \nvett \} $ base di V  
\begin{itemize}
\item lineare.\\
Sia $v=\ncomb $ e $w=b_1v_1+ \cdots + b_n v_n $ allora
$$ [ v+ w]_{\B} =\begin{pmatrix}
a_1+ b_1 \\ \vdots \\ a_n+  b_n 
\end{pmatrix}= \begin{pmatrix}a_1\\ \vdots \\ a_n 
\end{pmatrix} + \begin{pmatrix}
b_1\\ \vdots \\ b_n 
\end{pmatrix} = [v]_{\B} + [w]_{\B} $$


$$ [ \lambda v]_{\B} =\begin{pmatrix}
\lambda a_1 \\ \vdots \\ \lambda a_n  
\end{pmatrix}= \lambda \begin{pmatrix}a_1\\ \vdots \\ a_n 
\end{pmatrix} =\lambda [v]_{\B}  $$
\item iniettiva.\\
$$Ker [\, ]_{\B} = \left\{ v \in V \, \left| \,[v]_{\B} = \begin{pmatrix}
0 \\ \vdots \\ 0
\end{pmatrix} \right. \right\} =\{ 0 \} $$
\item suriettiva.\\
$$ \forall \begin{pmatrix}
a_1 \\ \vdots \\ a_n
\end{pmatrix} \in \K ^ n  \quad \exists v= \ncomb \quad [v]_{\B}=\begin{pmatrix}
a_1 \\ \vdots \\ a_n
\end{pmatrix}$$
\end{itemize}

\end{prop}

\newpage
\section{Matrice associata ad un'applicazione lineare}
\begin{defn}[Matrice associata a $f$ rispetto a $\B $ e  $\D$]\bianco
Data $f \in Hom(V,W)$ e $\B$ e $\D$ basi rispettivamente di $V$ e $W$ \`e definita un unica applicazione lineare $$ \AB(f) :\, \K^n \to \K^n $$ che fa commutare il seguente diagramma
$$ \begin{tikzcd} V \arrow[r,"f"]
\arrow{d}{[\, ]_\B} & V \arrow{d}{[\,]_\D} \\ \K^n \arrow{r}{\AB(f)} & \K^m
\end{tikzcd}$$
\begin{oss}\label{mat_appl} Se $\B = \{ \nvett \} $ allora

$$ \AB (f) = \left( [f(v_1)]_\D , \dots , [f(v_n)]_\D  \right) $$
infatti poich\`e il diagramma commuta 
 
Inoltre discende dalla definizione che

$$ \forall v \in V \qquad \left[ f(v) \right]_\D = \AB (f) \cdot \left[ v \right]_\B $$
\end{oss}
\end{defn}
\spazio
\begin{prop}[Matrice associata alla composizione]\label{mat_comp}\bianco
Siano $f\in Hom(V,W) $ e $g \in Hom(W,Z)$, dette $\B $, $\D$ e $\RR$  basi rispettivamente di $V$, $W$ e $Z$ segue che 
$$ M^\B_\RR (g\circ f ) = M^\D_\RR (g) \cdot \AB (f)$$
\proof
$\forall v \in V $
$$ [ g(f(v)]_\RR = M_\RR^\D (g) \cdot [ f(v)]_\D = M_\RR^\D(g) \cdot \AB (f) \cdot [v]_\B $$
Dove i passaggi sono giustificati dall'osservazione precedente.\\
Ora poich\`e vale $\forall v$ abbiamo l'uguaglianza voluta
\endproof
\begin{oss}Grazie alla proposizione sopra enunciata otteniamo che il seguente diagramma commuta

$$\begin{tikzcd} V \arrow[r,"f"] \arrow[bend left]{rr}{g\circ f} \arrow{d}{[\,]_\B } &W \arrow[r,"g"] \arrow{d}{[\, ]_\D }  & Z \arrow{d}{[\,]_\RR }\\ \K^n \arrow[r, "A" ] \arrow[bend right]{rr}{B\cdot A } & \K^m \arrow[r,"B"] & \K^q 
\end{tikzcd}$$
dove 
$$ A = \AB (f) \qquad  B =M^\D_\RR (g)$$

\end{oss}
\end{prop}
\newpage
Grazie alle matrici associate ad un applicazione lineare possiamo  riprendere quanto detto in \ref{Matrici_Come_Lineari} e esplicitare l'isomorfismo tra $Hom(\K^n , \K^m ) $ e $M(m,n,\K)$
\begin{thm}[$ \AB $ \`e un isomorfismo ] 
Siano $V$ e $W$ spazi vettoriali 
Sia $ \B =\{ \nvett \} $ una base di $V$ e  $ \D =\{ w_1, \, \dots , \, w_m \} $ una base di $W$.\\
Allora l'applicazione
$$ \AB :\, Hom (V ,W ) \to M ( m,n, \K ) \qquad f \to \AB (f) $$
\`e un isomorfismo
\proof
Mostriamo che $f$ \`e
\begin{itemize}
\item lineare.\\
Siano $f,g \in Hom (V,W ) $ allora
$$ \AB ( f+g) = ( [ (f+g)(v_1) ]_{\D} , \, \dots , \,  [ (f+g)(v_n) ]_{\D} ) =$$
$$  ( [ f(v_1)+g(v_1) ]_{\D} , \, \dots , \,  [ f(v_n) + g (v_n) ]_{\D} ) =
$$
$$ = ( [f(v_1)]_{\D} , \, \dots , \, [f(v_n)]_{\D} )\quad + \quad ( [g(v_1)]_{\D} , \, \dots , \, [g(v_n)]_{\D} )= \AB ( f) + \AB (g) $$
\item Iniettiva.\\
Se $f \in Ker \AB $ allora
$$ f(v_1)= \cdots = f(v_n ) = 0 $$ 
Ora poich\`e abbiamo definito $f$ su una base esse \`e ben definita ed \`e l'applicazione nulla, il nucleo, dunque, \`e ridotto al solo $0$ di $Hom(V,W)$
\item Suriettiva.\\
$\forall A \in M(m,n,\K)$ possiamo considerare una $f$ che faccia commutare il diagramma 
$$ \begin{tikzcd} V \arrow[r,"f"] \arrow{d}{[\, ]_\B} & V \arrow{d}{[\,]_\D} \\ \K^n \arrow[r,"A"] & \K^m
\end{tikzcd} $$
tale $f$ esiste poich\`e $[\,]_\B$ e $[\,]_\D$ sono isomorfismi dunque invertibili.\\
Dalla definizione data di matrice associata segue che $A=\AB (f) $
\end{itemize}
\endproof
\end{thm}
\spazio
\begin{cor}$$ End (V) \cong M(n,\K)$$
$$ GL(V)\cong GL(n,\K)$$
\proof
Prendiamo in entrambi i casi come base in partenza ed in arriva la stessa base $\B$ di V dunque le matrici sono quadrate.\\
Inoltre per dimostrare il secondo isomorfismo, osserviamo che
$$ f\in GL(V) \quad \implica \quad \exists f^{-1} \in GL(V)$$ 
ora per quanto visto sulla composizione di funzioni
$$M_\B^\B (f \circ f^{-1}) = M_\B^\B (f) \cdot M_\B^\B(f^{-1} )$$
Ma $$M_\B^B (f\circ f^{-1}) = M_\B^\B (id) = I_n  \quad \implica \quad 
 \left( M_\B^\B(f) \right)^{-1} = M_\B^\B (f^{-1})$$
\endproof \end{cor}
\newpage

\subsection{Matrice cambiamento di base}
\begin{defn}[Matrice cambiamento di base] \bianco
Siano $\D$ e $\B$ basi di $V$.\\
Definiamo matrice del cambiamento di base da $\D$ a $\B$ la matrice 
$$ \AB (id_V) $$
\end{defn}
\spazio
\begin{lem}Sia 
$$ g:\, V \to \K^n $$ 
Allora  $$\exists ! \B \text{ base di } V \tc g = [\, ]_\B $$
\proof
Se una tale $\B$ esiste allora poich\`e $[\, ]_\B$ \`e invertibile anche $g$ lo \`e dunque 
$$ \B = \left\{ g^{-1}(e_1), \dots, g^{-1}(e_n) \right\}$$
Osserviamo, inoltre, che una tale base soddisfa le richieste.
\endproof
\end{lem}

\begin{prop} Sia $V $ uno spazio vettoriale  e sia $\B$ una sua base.\\
Sia $A \in GL(n) $ . Allora
\begin{itemize}
\item[(i)] $ \exists ! \D $ base di V tale che $ A = \AB (id) $
\item[(ii)] $ \exists ! \mathfrak{T}$ base di V tale che $ A =M_{\B}\alla { \mathfrak{T}}(id)$

\end{itemize}
\proof $(i) $ Le ipotesi creano un diagramma del genere
$$\begin{tikzcd} 
V
\arrow[r,"id"]
\arrow{d}{[\,]_\B } 
&V
\\
K^n 
\arrow[r,"A"]
&K^n
\end{tikzcd}$$
Ponendo $g=A \circ [\, ]_\B $ segue che il diagramma  sottostante commuta 
$$\begin{tikzcd} 
V
\arrow[r,"id"]
\arrow{d}{[\,]_\B}
&V
\arrow[d,"g"]\\
K^n 
\arrow[r,"A"]
&K^n
\end{tikzcd}$$
Ora concludo applicando il lemma precedente
\endproof
\begin{oss}La proposizione precedente ci dice che ogni matrice invertibile pu\'o essere interpretata come
\begin{itemize}
\item Un endomorfismo
\item Una matrice di cambiamento di base (in avanti o in indietro)
\end{itemize}
\end{oss}
\end{prop}
\newpage
\section{SD-equivalenza}
\begin{defn}[SD-equivalenza funzioni]
Siano $f,g \in Hom (V,W)$ 
$$f \sim _{SD} g \quad \Leftrightarrow\quad \exists h \in GL(V),\quad \exists k \in GL(W) \quad g=k\circ f \circ h$$
\end{defn}
ed in versione matriciale 
\begin{defn}[SD-equivalenza matrici]
Siano $A,b \in M(m,\, n,\, \K) $ Allora
$$ A\sd B \quad \Leftrightarrow \quad \exists M \in GL(m) , \quad \exists N \in Gl(n) \quad B=MAN$$
\end{defn}
\begin{oss}
Le relazioni sopra definite sono di equivalenza 
\end{oss}

\spazio
\begin{prop}
I seguenti fatti sono equivalenti
\begin{itemize}
\item[(i)] $ f\sd g $
\item[(ii)] $\exists \B $ base di $V$ e $\exists \D$ base di $W$ tali che 
$$ \AB (f) \sd \AB(g)$$
\item[(iii)] $ \exists \B, \, \B' $ basi di $V$ e $\exists \D, \, \D'$ basi di $W$ tali che 
$$ M_{\D'}^{\B'} (f)= \AB (g)$$
\end{itemize}
\proof Dimostriamo le varie implicazioni
\begin{itemize}
\item (i) $\implica $ (ii).\\
Le ipotesi ci portano alla seguente situazione
$$ \begin{tikzcd} 
V \arrow[r,"k"] \arrow{d}{[\,]_\B } &V \arrow[r,"f"] \arrow{d}{[\,]_\B} & W \arrow[r,"h"] \arrow{d}{[\,]_\D }&W \arrow{d}{[\,]_\D }\\ \K^n \arrow[dashed,r,"N"] & \K^n \arrow[r,"A"] &\K^m \arrow[dashed, r,"M"]&\K^m
\end{tikzcd}$$
Ora poich\`e $ g = h \circ f \circ k $ ne segue per quanto detto in~\ref{mat_comp} che $\AB (g) = NAM $.\\
Ora 
$$ h\in GL(W) \quad \implica M \in GL(m,\K)$$
$$ k\in GL(V) \quad \implica M \in GL(n,\K)$$
da cui 
$$ \AB (g)= M \AB (f) N \quad \implica \quad \AB (g) \sd \AB(f)$$
\item (ii)  $\implica$ (i)\\
Le ipotesi ci portano alla seguente situazione
$$ \begin{tikzcd} 
V \arrow[dashed, r,"k"] \arrow{d}{[\,]_\B } &V \arrow[r,"f"] \arrow{d}{[\,]_\B} & W \arrow[dashed, r,"h"] \arrow{d}{[\,]_\D }&W \arrow{d}{[\,]_\D }\\ \K^n \arrow[r,"N"] & \K^n \arrow[r,"A"] &\K^m \arrow[r,"M"]&\K^m
\end{tikzcd}$$
dove $MAN=\AB (g)$ ora poich\`e 
$$ M\in GL(m,\K) \quad \implica h \in GL(W)$$
$$ N \in GL(n,\K) \quad \implica M \in GL(V)$$
da cui 
$$ g = h\circ f \circ k \quad \implica \quad f \sd g $$
\item  (i) $\implica$ (iii)\\
L'ipotesi ci porta ad un diagramma come segue 
$$ \begin{tikzcd} 
V \arrow[r,"k"] \arrow{d}{[\,]_\B } &V \arrow[r,"f"] \arrow{d}{[\,]_\B} & W \arrow[r,"h"] \arrow{d}{[\,]_\D }&W \arrow{d}{[\,]_\D }\\ \K^n \arrow[dashed,r,"N"] & \K^n \arrow[r,"A"] &\K^m \arrow[dashed, r,"M"]&\K^m
\end{tikzcd}$$
dove $\AB(g)=MAN $\\
Ora poich\`e $M$ e $N$ sono invertibili li posso interpretare come matrice di cambiamento di base quindi
$$ \exists \B' \text{ base di } V \tc M^{\B'}_\B (id_V) = N $$
$$ \exists \D' \text{ base di } W \tc M_{\D'}^\D(id_W) = M $$
otteniamo dunque
$$ \begin{tikzcd} 
V \arrow{r}{id_V} \arrow{d}{[\,]_{\B'} } &V \arrow[r,"f"] \arrow{d}{[\,]_\B} & W \arrow{r}{id_W} \arrow{d}{[\,]_\D }&W \arrow{d}{[\,]_{\D'} }\\ \K^n \arrow[r,"N"] & \K^n \arrow[r,"A"] &\K^m \arrow[r,"M"]&\K^m
\end{tikzcd}$$
Dunque 
$$ M^{\B'}_{\D'} (f) = M^{\B'}_{\D'} (id_W\circ f \circ id_V) =M A N =\AB(g)$$
\item (iii) $\implica$ (ii)\\
Sia $ N = M^{\B'}_\B $ e $M =M^\D_{\D'}$ allora  per quanto detto sopra
$$ M^{B'}_{\D'}(f)= N \AB (f) M $$
Ora poich\`e $N$ e $M$ sono matrici di cambiamento di basi sono invertibili dunque
$$  M^{B'}_{\D'}(f) \sd \AB (f)$$
Ma per ipotesi $ M^{B'}_{\D'}(f)=\AB (g)$ dunque
$$ \AB (f) \sd \AB(g)$$
\end{itemize}
\endproof
\end{prop}
\newpage
Per continuare a studiare la relazione \`e utile la seguente definizione
\begin{defn}[Rango]\bianco
Sia $f \in Hom (V,W)$, allora definiamo il rango di f come 
$$rk( f) = \dim Im \, f $$
\end{defn}

\spazio

\begin{prop}[Invariante completo per $\sd$]
$$ f\sd g \quad \ses \quad rk (f) =rk(g)  $$
\proof $\implica$ $g=k ]\circ f \circ h $ con $k$ e $h$ isomorfismi.
Poich\`e applicazioni lineari mandano sottospazi in sottospazi e  gli isomorfismi preservano la dimensione si conclude che 
$$ \dim Im f = \dim Im g$$
$\Leftarrow$ Ripercorriamo quanto fatto nella dimostrazione della formula delle dimensioni di nucleo e immagine (vedi~\ref{dim_ker_imm})\\
Siano $f, g \in Hom(V,W) $ con $\dim V = n $ e $\dim W =m $ e sia $ rk(f)=rk(g)=r$.\\
Sia $$\{  v_{r+1}, \, \dots , \, v_n \} \text{ una base del nucleo di } f $$
Estendiamolo a $$\B= \{ v_1 \, \dots , \, v_r, \, v_{r+1}, \, \dots \, v_n \} \text{ base di } V$$
Da fatti noti sappiamo che  $$ \{ f(v_1), \dots, f(v_r)  \} \text{ \`e una base  dell'immagine di} f $$
Estendiamola tale base a 
$$\D=\{ f(v_1), \dots , f(v_r) , \, w_{r+1}, \, \cdots , \, w_m \} \text{  base di } W$$
Per come sono state costruite le basi risulta che
$$\AB (f)= \begin{pmatrix}
I_r & 0 \\
0 & 0 
\end{pmatrix}$$
Se ripercorriamo la costruzione, considerando $g$, otteniamo 2 basi $\B' $ (base di $V$) e $\D' $ (base di $W$) tale che   
$$ M_{\D'}\alla { \B ' }(g)= \begin{pmatrix}
I_r & 0 \\
0 & 0 
\end{pmatrix}$$
Da ci\'o segue che
$$\AB (f)=M_{\D'}\alla { \B ' }(g)\quad \implica \quad f\sd g $$
\endproof

\begin{oss}
In ogni classe SD-equivalenza possiamo scegliere un rappresentante in forma normale $ J_r (m,n)$ dove r è il rango.\\
$$
J_r=
\left(
\begin{array}{c|c}
I_r & 0 \\
\hline
0 & 0
\end{array}
\right)
$$

\end{oss}
\end{prop}
\newpage
\begin{lem}
$$ M \in GL(n,\K) \quad \implica \quad M^t \in GL(n,\K)$$
\proof
Essendo $M$ invertibile 
$$ \exists M^{-1}\in GL(n.\K) \tc M M^{-1} =I_n $$ 
Ora applicando la trasposta e ricordando che $(AB)^t = B^t A^t $ otteniamo 
$$ \left( M^{-1} \right)^t M^t = I_n^t =I_n$$ 
ora se consideriamo anche $M^{-1}M=I_n$ e data l'unicit\'a dell'inversa otteniamo
$$ \forall M \in GL(n, \K) \quad \left( M^t \right)^{-1}= \left(M^{-1}\right)^t \quad \implica \quad M^t \in GL(n,\K)$$
\endproof
\end{lem}
\begin{cor}
$$ rk (A) = rk \left( A^t \right)$$
\proof
Sia $A \in M(m,n,\K )$ e sia $rK(A)=r $ .\\
Poich\`e il rango \`e un invariante completo per $\sd$ allora  
$$ A \sd J_r \quad \implica \quad  
 \exists M , \, N \text{ invertibili } \tc A = M\cdot J_r \cdot N  $$
Dunque se consideriamo 
$$  A^t  = N^t  \cdot J_r^t  \cdot  M^t  = N^t  \cdot  J_r \cdot   M^t   $$ 
Ora per il lemma precedente $ M^t$ e $N^t $ sono invertibili  dunque
 $$  A^t \sd J_r \quad \implica \quad  rk\left( A^t \right)=r=rk(A)$$
\endproof
\end{cor}
\newpage
\section{Sistemi lineari e algoritmo di Gauss}
\subsection{Sistemi lineari}
\begin{defn}
Definiamo il sistema lineare di m equazione in n incognite
$$ \begin{cases}
a_{11} x_1 + \, \cdots \, + a_{1n}x_n=b_1 \\
\vdots \\ 
a_{m1} x_1 + \, \cdots \, + a_{mn}x_n=b_m 
\end{cases}$$
\end{defn}
\begin{oss}
Il sistema lineare pu\'o essere scritto nella forma $AX=B $ dove
$$ A= \begin{pmatrix}
a_{11} & \cdots & a_{1n}\\
\vdots & \ddots & \vdots \\
a_{m1} & \cdots & a_{mn}
\end{pmatrix} \in M( m,n , \K ) , $$ 

$$ X = \begin{pmatrix} x_1 \\ \vdots \\ x_n \end{pmatrix} \in \K^n ,\qquad B=\begin{pmatrix}
b_1\\ \vdots \\ b_m 
\end{pmatrix} \in \K \alla m $$
\end{oss}
\spazio
\begin{defn} Se $B=0$ il sistema si dice omogeneo.\\
Le soluzioni del sistema omogeneo sono
$$ \{ X \in \K ^n \quad \vert \quad AX=0 \} = Ker(A)$$
\end{defn}
\begin{defn} Sia $AX=B$.\\
Il sistema omogeneo associato \`e il sistema $AX=0$
\end{defn}

\spazio
\begin{prop}Dato un sistema $AX=B$.\\
Se il sistema ha soluzione, presa una particolare $Y_B$ allora 
$$ Sol_B= \{ Y_B + X \, \vert X \in Sol_0 \} $$
\proof $\supseteq$ Sia $X \in Sol_0 $  
$$ Y_B+X \in Sol_B \quad \ses \quad A ( Y_B + X )=B \quad \ses \quad AY_B +AX=B+0=B$$
L'ultima implicazione \`e vera, dunque anche la prima
\proof $\subseteq$ Sia $X \in Sol_B $.\\
$$X=Y_B + ( X- Y_B) \quad X-Y_B\in Sol_0 $$
infatti $$ A ( X-Y_B)= AX- AY_B=B-B=0 $$
\endproof
\end{prop}
\subsection{Algoritmo di Gauss}

L'algoritmo agisce o sulle righe o sulle colonne, qui viene descritto sulle righe ma per le colonne é del tutto analogo.\\
L'algoritmo prende in input una matrice $ A \in M(m,n, \K) $ e ne restituisce un'altra denotata con $\hat{A}_R \in M(m,n,\K)$ , il pedice indica che la matrice è ottenuta mediante le righe.\\
L'algoritmo costa di operazioni elementari sulle righe, distinte in 3 tipi:
\begin{itemize}
    \item [I)] Scambia tra loro 2 righe $A_i \leftrightarrow  A_J$
    \item[II)] Moltiplica una riga per uno scalare $ \neq 0 $  $ A_i \rightarrow c A_i \quad c \neq 0$
    \item[III)]	Somma ad una riga un multiplo di un altra $ A_i \rightarrow A_i + c A_j$
\end{itemize}
\spazio
Specifichiamo come funziona l'algoritmo:\\
Indichiamo con $A^1, \, \cdots \, A^n$ le colonne di A.\\

Se $ A=0 $ allora $\hat{A}_R=0$ e l'algoritmo si interrompe.\\
Se $ A \neq 0 $ allora esiste una colonna non nulla, si consideri il più piccolo indice $j$ tale che $A_j \neq 0 $ .\\
Sia $i$ il più piccolo indice di riga tale che $a_{ij}\neq 0 $ in questo caso $A_i\leftrightarrow A_j$\\
Abbiamo ottenuto una matrice del genere 
$$\begin{pmatrix}
0 & \cdots & 0 & 1 & \cdots\\
\vdots & & \vdots & ? & \vdots\\
0 & \cdots & 0 & ? & \vdots
\end{pmatrix}$$Applichiamo operazione del secondo tipo in modo che sotto il primo 1 ci siano solamente 0 
$$\begin{pmatrix}
0 & \cdots & 0 & 1 & \cdots\\
\vdots & & \vdots & \vdots & \vdots\\
0 & \cdots & 0 & 0 & \vdots
\end{pmatrix}$$A questo punto consideriamo la matrice $ \tilde{A}$ ottenuta dimenticando la prima riga, e applichiamo l'algoritmo finché é possibile.\\
Iteriamo il procedimento e otteniamo una matrice $\overline{A}_R$ (simile a questa) detta matrici a scalini
$$\begin{pmatrix}
0 & 1 &  ? & ?  &? & ? \\
0 & 0 & 0 & 1 &? & ?\\
0 & 0 & 0 & 0 & 0& 1\\
0 & 0 & 0 & 0 & 0& 0 \\
\end{pmatrix}$$
Ora continuando ad applicare operazione del 3 tipo si riesce ad ottenere degli zeri sopra gli $1$, ottenendo così la matrice $\hat{A}_R $ detta a scalini completi
$$\begin{pmatrix}
0 & 1 &  ? & 0  &? & 0 \\
0 & 0 & 0 & 1 &? & 0\\
0 & 0 & 0 & 0 & 0& 1\\
0 & 0 & 0 & 0 & 0& 0 \\
\end{pmatrix}$$
Gli $1$ vengono chiamati PIVOT di $\hat{A}_R$ .\\
La matrice $\hat{A}_R$ ha degli 0, sotto, sopra e destra dei pivot .
\spazio
\subsubsection{Calcolo del rango}
\begin{defn}[Matrice R-elementare]\bianco
Sia F una matrice di taglia $ n \times n $.\\
 F si dice  R-elementare se si ottiene applicando un' operazione R-elementare alla matrice identica $I_n$
\end{defn}

\begin{lem}
Sia A una matrice $ m \times n $ e sia $ \omega $ un'operazione elementare di un dato tipo.

$$A\xrightarrow{\omega}A'$$
$$ I_m \xrightarrow {\omega} F_\omega $$Allora $$A'=F_\omega A$$

\end{lem}\spazio
\begin{lem}
Ogni matrice R-elementare è invertibile e la sua inversa è elementare dello stesso tipo
\end{lem}

\begin{oss}
L'algoritmo applica una serie di operazione elementari quindi :
$$ \hat{A}_R=F_k \cdots F_1 A $$ ma $F_k ,\cdots, F_1 \in GL (m,\K) $ quindi anche il loro prodotto appartiene al gruppo lineare ne segue che 
$$ \hat{A}_R \sim_S A$$ %e trasponendo $$\hat{A}_C \sim_D A $$
\end{oss}

\spazio
\begin{prop}[Rango] $$rk(A)=rk(\hat{A} )$$
\proof 
Sia $ A \in M(m,n, \K) $ .
Essendo $ \hat{A}_R \sim_S A$ allora $ rango \, A = rango \, \hat{A}_R$ infatti possiamo vedere la relazione $\sim_S$ come un caso particolare di $\sd$\\
\begin{oss} 
L'algoritmo di Gauss permette di calcolare facilmente il rango di una matrice, conoscendo la sua ridotta a scalini, infatti il rango di una matrice a scalini \`e dato dal numero di pivot.
\end{oss}
\end{prop}
\newpage
\subsubsection{Sistema omogeneo}
\begin{prop}[Nucleo] $$\ker A= \ker \hat{A}_R$$ 
\proof
Se $ A \sim_S D $ allora vale che $D=QA$.
Mostriamo che valgono entrambe le inclusioni.
\begin{itemize}
\item Sia $x \in \ker A $
$$ DX= (QA)X=Q(0)=0  \text{ ovvero }  \ker A  \subseteq \ker D $$
\item 
Sia $ x \in \ker D  $
$$  DX=(QA)X = 0 \quad Q(A(X))=0 $$  ma $ Q \in GL$ quindi ammette inversa  $$ AX=0 \text{ ovvero } \ker D  \subseteq \ker A $$
\end{itemize}
Valgono entrambe le disuguaglianze dunque $\ker A =\ker D$
\endproof
\end{prop}
\begin{oss}[Equazione omogenea]
Dalla proposizione osserviamo che invece che risolvere l'equazione
$ AX=0$ possiamo risolvere $ \hat{A}_R X=0$
\end{oss}
\subsubsection{Sistema non omogeneo}
Occupiamoci ora del sistema non $ AX=B$ con $ B \neq 0$ quindi non omogeneo
\begin{defn}[Matrice completa del sistema]
Consideriamo la matrice 
$$ 
\begin{pmatrix}
A  & \vert  & B 
\end{pmatrix}$$
tale matrice viene chiamata matrice completa del sistema e si ottiene dalla matrice A dei coefficienti aggiungendo la colonna B dei termini noti.
\end{defn}
\spazio 
\begin{prop} [Principio di Rouché - Capelli]\bianco % dubbio di logica
Il sistema $ AX=B$ ha soluzione se e solo se 
$$ rk \begin{pmatrix}
A
\end{pmatrix}= rk \begin{pmatrix}
A &  \vert & B 
\end{pmatrix}
$$
\proof $ \implica$ \\
Sia  $ A = ( A^1 \cdots A^n )$.
Supponiamo che il sistema abbia soluzione dunque
$$ \exists \begin{pmatrix} x_1 \\ \vdots \\ x_n  \end{pmatrix}  \in \K^n\tc B = x_1 A^1 + \cdots x_n A^n$$ dunque
$$Span ( A^1, \cdots , A^n ) = Span ( A^1, \cdots , A^n , B )  \quad \implica \quad rk \begin{pmatrix}
A
\end{pmatrix}= rk \begin{pmatrix}
A &  \vert & B 
\end{pmatrix}$$
 $\Leftarrow$ in modo contro-nominale.\\
Supponiamo che il sistema non abbia soluzione quindi $ B \neq AX \quad \forall X\in \K ^n $ da cui
$$ Span ( A^1, \, \cdots \, A^n ) \subseteq Span ( A^1 , \, \cdots , \, B ) \implica rk \begin{pmatrix}
A
\end{pmatrix}<  rk \begin{pmatrix}
A &  \vert & B 
\end{pmatrix}$$
\endproof 
\end{prop}
\subsubsection{Calcolo dell'inversa}
\begin{prop}[Matrici invertibili]
Sia A una matrice di taglia $ n \times n$ 
$$ A \in GL(n, \mathbb{K}) \quad \ses \quad rk(A)= n $$
\proof $\implica$\\
Se $ A \in GL(n) $ allora $\exists Q \in  GL(n) $ tale che $QA=I_n$ quindi
$$ A \sim_S I_n \implica rk(A)=rk(I_n)=n $$
\proof $\Leftarrow$\\
Se $ rk(A)=n $ allora
 $$A\sim_S I_n \implica \exists Q \in GL(n) \tc QA=I \implica A \in GL(n)$$
\endproof
\begin{oss}
Per trovare la matrice inversa basta tenere conto delle operazione R-elementari per passare dalla matrice $ A $ a $I_n$ ovvero 
$$ \begin{pmatrix}
A & \vert &I_n 
\end{pmatrix} \xrightarrow{R-Gauss} \begin{pmatrix}
I_n &\vert &  Q
\end{pmatrix}
$$
infatti se $A$ viene trasformata con R-Gauss in $I_n$ 
$$ I_n= F_1 \cdots F_n \cdot A \text { con }  F_i \text{  R-elementari} $$
Pongo $$Q =A^{-1}=  F_1 \cdots F_n$$
\end{oss}
\end{prop}
\begin{cor} Ogni matrice invertibile \`e prodotto di matrici R-elementare
\end{cor}
\spazio
\subsubsection{Vettori linearmente indipendenti}
\begin{prop} Estrazione di una base da un gruppo di generatori.\\
Dati $\nvett \in \K ^n $, sia $ A= ( v_1 \, \vert \, \cdots \, \vert \, v_n )$.\\
Detta S una ridotta a scalini di A, se $ S^{a_1} , \, \cdots \, , S\alla{ a_r} $ sono le colonne dove sono presenti i pivot allora $ \{ v_{a_1}, \, \cdots \, ,\, v_{a_r} \} $ sono una base di $Span( \nvett ) $
\end{prop}
\newpage
\section{D-equivalenza}
\begin{defn}[D-equivalenza]\bianco
Siano $A,B \in M(m,n,\K) $ allora
$$ A \sim_D B \quad \ses \quad \exists P \in GL(n) \quad B=AP$$
\end{defn}
\begin{oss} Lo spazio generato dalle colonne \`e invariante, quindi se $A
\sim_D B $ allora $Im(A)=Im(B) $ da cui anche $rk(A)=rk(B) $
\end{oss}

\begin{defn}
$\forall  0\leq r \leq \min(m,n) $ allora definiamo 
$$ M_r(m,n, \K ) = \{ A \in M(m,n,\K) \, \vert \, rk(A)=r \} $$
\end{defn}
Lo spazio $M(m,n,\K) $ \`e l'unione disgiunta degli insiemi $M_r(m,n,\K) $ al variare del rango $r$, possiamo restringere la relazione $\sim_D$ a ciascuna di essi.\\
Per semplicit\'a ci restringiamo nel caso di rango massimo ovvero $ r=\min(m,n) $ 

\spazio


 \subsection*{Regime suriettivo} 


$$\begin{tikzcd} \K^n\arrow[r,"A"] & \K ^m \end{tikzcd} $$
In questo caso $Im(A)= \K ^m $ da cui $rk(A)=m $.\\
Essendo la funzione suriettiva vale che $n \geq m $ 	quindi se $rk(A)=m$ \`e il massimo possibile . 
\spazio
Considero $ \hat{A}_C$ dato che il rango \`e m ottengo che 
$$ \hat{A}_C=J_m(m,n)= \begin{pmatrix}
I_m & \vert & 0_{m,n-m} 
\end{pmatrix}$$
Ne segue che il quoziente $M_m(m,n,\K)\backslash\sim_D$ \`e ridotto ad un solo punto e quindi $J_m(m,n)$ \`e il rappresentante normale dell'unica classe di equivalenza
 \newpage
 \subsection*{Regime iniettivo} 
 $$\begin{tikzcd} \K^n\arrow[r,"A"] & \K ^m \end{tikzcd} $$
 In questo caso $Im(A)=\K^n $ da cui $rk(A)=n $.\\
 Essendo la funzione iniettiva vale che $n\leq m $
 \begin{defn}[Simbolo di Schubert]\bianco
 Sia $\hat{A}_C$ la matrice ottenuta applicando ad $A$ l'algoritmo di Gauss (completo) rispetto alle colonne.\\
Se il rango di $A$ \`e n la matrice a scalini avr\'a $n$ pivot.
 $$s(\hat{A}_C)=(s_1,\, \cdots\, ,\, s_n )$$
 dove $s_j$ \`e uguale all'indice di riga del j-esimo pivot di $\hat{A}_C$
 \end{defn}
 
 Poniamo $\forall j=0,\, \cdots \, , \, m $
 $$ p_j:\, \K^m \to \K^{m-j} \qquad  p_j \begin{pmatrix}
 x_1 \\ \vdots \\ x_m
 \end{pmatrix} = \begin{pmatrix}
 x_1 \\ \vdots \\ x_{m-j} 
 \end{pmatrix}$$
 Poniamo inoltre
 $$ d_j=\dim(p_j(Im(A))$$
 \begin{prop} La dimensione $d_j$ varia da $n$ a $0$ in modo monotono diminuendo di 1, passando da $d_{s_i+1} $ a $d_{s_i}$ .\\
 Da ci\'o segue che il simbolo dipende solo da $Im(A)$ dunque pu\'o essere definito anche 
 $$s(A)=s(Im(A)) \quad \forall A \in M_n(m,n,\K) $$
 Il simbolo resta costante, dunque sulla classe di equivalenza .
\proof Segue immediatamente dalla forma della matrice a scalini
 \end{prop}
 \spazio
 Poniamo $\forall s $ simbolo 
 $$M_{n,s}(m,n,\K) = \{ A \in M_n(m,n,\K) \, \vert \, s(A)=s \} $$ 
 \begin{prop} Fissato un simbolo $s$.\\
 Siano $A,B \in M_{n,s}(m,n,\K) $.
 $$ Im(A)=Im(B) \quad \implica \quad \hat{A}_C= \hat{B}_C $$
 \proof
 Poniamo $L=Im(A)$.\\ 
 Per la proposizione precedente segue che $s=s(L)=s(A)=s(\hat{A}_C)$.\\
 Sia 
 $$ p_s :\, \K ^m \to \K^n \qquad p_s(X)= \begin{pmatrix}
 x_{s_1} \\ \vdots \\ x_{s_n} 
 \end{pmatrix}$$ 
 \`e facile verificare che la restrizione di $p_s$ ad $L$ \`e un isomorfismo.\\
 Le colonne di $\hat{A}_C$ formano l'unica base di L che viene mandata dalla restrizione nella base canonica di $\K^n $; da ci\'o segue che $\hat{A}_C $ \`e completamente determinato da $L $ da cui segue la proposizione.
 \endproof
 \end{prop}
 \spazio
Mettendo insieme quanto detto fino ad ora
\begin{cor}[Invariante completo]\bianco
Siano $A,B \in M(m,n,\K) $ allora
$$ A \sim_D B \quad \ses \quad Im(A)=Im(B)$$
Per ogni $A$ come sopra, $\hat{A}_C$ \`e il rappresentante in forma normale della classe di equivalenza $[A]_D$
\end{cor}

\spazio
Possiamo riformulare quanto detto sopra.
\begin{defn}[Insieme di Grassman]
$$G_{m,n} =\{ L \in \K ^m \, \vert \, \dim L=n \}$$
\end{defn}

L'applicazione
$$ \pi :\, M_n(m,n,\K) \to G_{n,m}  \qquad \pi(A)= Im(A)$$
\`e suriettiva, inoltre $ \pi(A)=\pi(B) $ se e solo se $ A \sim_D B $ quindi l'insieme di Grassman pu\'o essere identificato con il quoziente per la relazione $\sim_D$ 
$$ G_{m,n} =M(m,n,\K) \sbarra \sim_D$$
dunque $\pi$ si identifica come la proiezione naturale al quoziente 
\spazio

\begin{defn} Per ognuno dei $ { m \choose n } $ simboli $s$
$$ B_s=\{ L \in G_{m,n} \, \vert \, s(L)=s \} $$
$$\mathcal{A}_{C,s}=\{ \hat{A}_C \in  M_n(m,n,\K) \, \vert \, s(\hat{A}_C)=s $$
\end{defn}
\begin{oss} $G_{m,n} $ \`e unione disgiunta dei $B_s$ 
\end{oss}

Esplicitiamo la struttura di $\mathcal{A}_{C,s}$ 
\begin{defn} Sia $J_s$ l' unica matrice in $\mathcal{A}_{C,s}$ le cui entrate diverse dai pivot sono uguali a zero.
\end{defn}

Sia 
$$ \phi _s : \, \mathcal{A}_{C,s} \to M(m,n,\K) \qquad \phi_s (\hat{A}_C)=\hat{A}_C -J_s$$ 
Inoltre sia $\mathcal{V}_s=Im(\phi_s)$.\\
Chiaramente se
$$ \phi _s : \, \mathcal{A}_{C,s} \to \mathcal{V}_s$$
\`e biettivo 

\newpage
\begin{prop} Per ogni simbolo $s=(s_1,\, \cdots \, , \, s_n $, $\mathcal{V}_s$ \`e un sottospazio vettoriale di $M(m,n,\K) $ di dimensione
$$ d_s=\dim \mathcal{V}_s= \sum_{j=1}^n ( m-s_j-(n-j))$$
\proof Ogni matrice di $\mathcal{V}_s$ \`e caratterizzata da avere un pacchetto di entrate necessariamente nulle che dipendono dal simbolo, le altre entrate sono libere, la formula ha per j-esimo addendo il numero di parametri liberi sulla j-esima colonna.
\endproof
\end{prop}
$\mathcal{V}_s$ \`e detto cella di Schubert di $G_{m,n}$ di simbolo $s$ e dimensione $d_s$  ed \`e uno spazio affine $  \mathcal{A}_{C,s} = J_s+  \mathcal{V}_s$
\begin{oss} Alcune osservazioni sulle celle
\begin{enumerate}
\item La dimensione massima \`e $d_{max}=n(m-n) $ corrispondente al simbolo $s_{max}=(1,2,3,\, \cdots, n )$
\item La dimensione minima \`e $d_{min}=0$ corrispondente al simbolo $s_{min}=(m-n+1, \, \cdots, n )$
\end{enumerate}
\end{oss}
\spazio 
\spazio
\section{S-equivalenza}
Trasponendo e sostituendo ovunque "colonna" con "riga" abbiamo un trattamento "duale" della relazione per cui $A\sim_S B$ in particolare abbiamo
\begin{prop}
$$A \sim_S B \quad \ses \quad Ker(A)=Ker(B) $$
Per ogni $A$, $\hat{A}_R$ \`e il rappresentante in forma normale di $[A]_S$.\\
Nel regime iniettivo il quoziente \`e ridotto ad un solo punto e $J_{s_max} $ \`e il rappresentante in forma normale dell'unica classe di equivalenza.\\
Nel regime suriettivo il quoziente si identifica con $G_{m,n-m}$
\end{prop}
\newpage
\section{Spazio duale}
\begin{defn}
Sia V uno spazio vettoriale su $\K$, si definisce spazio duale
$$ V \alla * = Hom(V,\K) $$
Gli elementi $\varphi \in V^*$ sono detti funzionali.
\end{defn}
\begin{prop} Sia $\dim V= n $ allora $\dim V^*=n $
$$ V^*=Hom(V,\K) \cong Hom(\K^n,\K) = M(1,n,\K)$$
dove il primo isomorfismo deriva tramite passaggio di coordinate rispetto ad una base arbitrale di $V$ da cui 
$$\dim V^*=\dim M(1,n,\K)=n$$
\endproof
\end{prop}
\begin{defn}[Base duale]\bianco
Fissiamo una base $\B=\{ \nvett \, \} $ di V.\\
Allora la base duale di $\B^*$ di $\B$  
$$ \B^*=\{ v_1^*, \, \cdots \, , \, v_n^* \, \} $$ 
gli elementi $v_j^*$ sono definiti dalla propriet\'a
$$ v_j ^*=\delta_{i,j}= \begin{cases} 1 \quad \text{ se } i=j \\
0 \quad \text{ se } i \neq j
\end{cases}$$
$\delta_{i,j} $ viene chiamato delta di Kronecker.
\end{defn}
\begin{prop} Mostriamo che la base duale \`e una base dello spazio duale
\proof Mostriamo che i funzionali sono linearmente indipendenti, se 
$$ a_1 v_1^*+ \cdots + a_n v_n ^* =0 $$
dove lo $0$ \`e inteso il funzionale identicamente nullo (manda a 0 gli elementi di una base)
$$ \forall j \quad ( a_1 v_1^*+ \cdots + a_n v_n ^* )(v_j)=(a_j v_j^*)v_j=a_j=0$$
Poich\`e i funzionali sono $n$ e sono linearmente indipendenti, formano una base
\endproof
\end{prop}
\label{isoDuale}
\begin{cor} $\forall \B $ base di $V$.
$$ \varphi_\B : \, V \to V^* \qquad \varphi_\B(v_i)=v_i^*$$
\`e un isomorfismo

\begin{oss} L'isomorfismo sopra definito non \`e canonico perch\`e dipende dalla scelta della base $\B$
\end{oss}
\end{cor}
\spazio
\begin{defn}[Bi-duale]\bianco 
Sia $V$ uno spazio vettoriale.
$$\bidu =Hom \left( V^*,\K \right) $$
\end{defn}
\begin{oss} Poich\`e essere isomorfi \`e una relazione di equivalenza e dato che, per il corollario precedente, uno spazio vettoriale \`e isomorfo al suo duale ne segue che 
$$ V \cong V^\star  \quad V^\star \cong \bidu \quad\implica \quad V \cong \bidu$$
\end{oss}
\begin{prop} L' applicazione 
$$ \phi: \, V \to \bidu$$
$$ v \to \varphi_v $$
dove
$$ \varphi_v:\, V^\star \to \K $$
$$ \psi \to \psi(v)$$
\begin{itemize}
\item[(i)]\`e un isomorfismo canonico
\item[(ii)] $\forall \B$ base di $V$ , $ \varphi_{\B^*} \circ \varphi_\B$
\end{itemize}
\proof  \bianco\begin{itemize}
\item[(i)]
\begin{itemize}


\item Mostriamo che $\forall v \in V $ la funzione $\varphi_v$ \`e lineare. 
$$ \forall\psi_1, \psi_2 \in V^\star $$ $$\varphi_v( \psi_1 + \psi_2)=(\psi_1+\psi_2)(v)=\psi_1(v)+ \psi_2(v)=\varphi_v(\psi_1) + \varphi_v(\psi_2) $$
Il prodotto per scalari \`e analoga
\item Mostriamo che $\phi $ \`e lineare
$$ \forall v_1, v_2 \in V $$
$$ \phi( v_1+v_2)=\varphi_{v_1+v_2}(\psi)=\psi(v_1+v_2)=\psi(v_1)+\psi(v_2) = \varphi_{v_1}(\psi)+\varphi_{v_2}(\psi)=\phi(v_1)+\phi(v_2)$$
Il prodotto per scalari \`e analoga
\item $\phi$ \`e isomorfismo.\\
Poich\`e $\dim V= \dim \bidu$ basta dimostrare l'iniettivit\'a di $\phi$
$$Ker \phi=\big\{ v \in V \, \vert \, \phi(v)=0_{ \left( V^* \right) ^* }  \big \} \implica \forall \psi \in V^* \quad \varphi_v ( \psi )=0 \implica \psi(v)=0$$
Da cui segue che il kernel \`e ridotto al solo 0 

Dai punti precedenti segue che $\phi$ \`e un isomorfismo canonico tra uno spazio ed il suo bi-duale
\end{itemize}
\item[(ii)]
Sia 
$$ \B= \{ \nvett \} \text{ una base di V } $$
Poich\`e una funzione \`e univocamente determinata dai valori che assume su una base, occorre dimostrare che
$$ \phi(v_i ) = \varphi_{\B^\star} \circ \varphi_\B (v_i)   \quad \forall i = 1, \dot n $$
$$ \phi(v_i) = \varphi_{v_i}$$
$$ \varphi_{\B^\star} \circ \varphi_\B (v_i)= \varphi_{\B^\star} (v_i \star )= \left( v_i ^\star \right)^\star$$
Ora poich\`e $\B^\star$ \`e una base di $V^\star $ basta provare che 
$$ \varphi_{v_i} (v_j^\star ) = \left( v_i ^\star \right)^\star (v_j^\star) \quad \forall j=1,\dots , n \quad \forall i =1,\dots, n $$
Ora vale l'uguaglianza infatti entrambe valgono $\delta_{i,j}$
\end{itemize}
\endproof 
\end{prop}
\newpage
\subsection{Annullatore e luogo di zeri}
\begin{defn}[Annullatore] \bianco
Sia $W$ un sottospazio vettoriale di $V$ 
$$ Ann(W)=\{ \psi \in V^* \, \vert \, \forall w \in W \, \psi(w)=0 \} $$
\end{defn}

\begin{prop} L'annullatore di $W$ \`e un sottospazio di $V^*$ di dimensione 
$$ \dim (Ann(W))=\dim V - \dim W $$
\proof  Mostriamo che \`e un sottospazio 
\begin{itemize}
\item il funzionale identicamente nullo, annulla tutti i vettori di $V$ quindi anche quelli di $W$
\item $\forall \psi, \, \varphi \in Ann(W) \quad \forall w \in W \, \psi (w)=\varphi (w)=0 $ quindi
$$ (\psi + \varphi)(w)= psi (w)+ \varphi (w)=0 $$
In modo analogo si mostra la chiusura rispetto al prodotto scalare
\end{itemize}
Mostriamo che vale la formula sulle dimensioni.\\
Sia $\dim V = n $ e $ \dim W = k $ con $ n \geq k $.\\
Sia 
$$\{ w_1,\, \dots , \, w_k\} \text{  una base di } $$ estendiamola a 
$$ \{ w_1,\, \dots , \, w_k,\, v_{k+1} ,\, \dots ,\, v_n \} \text{ base di } V$$

L'insieme $$ \{v_{k+1}^* ,\, \dots ,\, v_n ^*\}$$ \`e una base dell'annullatore.\\
Poich'\`e tali vettori appartengono alla base duale, sono linearmente indipendenti, mostriamo che generano $Ann(W)$\\
$ \forall f \in Ann(W) $ poich\`e $\B^\star$ \`e una base dello spazio duale 
$$f= a_1 w_1^* + \dots + a_k w_k^* + a_{k+1} v_{k+1} ^* + \dots + a_n v_n ^* $$
Ora 
$$ f \in Ann (W) \quad \implica \quad f(w)=0 \quad \forall w \in W \quad \implica \quad f(w_i)=0 \quad \forall i=1,\dots k $$
quindi
$$ f(w_i) = (  a_1 w_1^* + \dots + a_k w_k^* + a_{k+1} v_{k+1} ^* + \dots + a_n v_n ^*) w_i = a_i =0 $$
Da cui segue che 
$$ \forall f \in Ann(W) \quad f = a_1 v_{k+1} + \dots + a_n v_n $$
\endproof
\end{prop}

\begin{defn}[Luogo di zeri]\bianco
Sia $ U$ un sottospazio di $V^*$ allora
$$Z(U)=\{ v \in V \, \vert \, \forall \psi \in U \quad \psi (v)=0 \} $$
\end{defn}
Posso vedere il luogo di zeri come una funzione
$$ Z: \, G_{n-k} (V^*) \to G_k(V) \quad U \to Z(U) $$
\begin{prop}
Il luogo di zeri \`e un sottospazio di $V$, la dimostrazione \`e analoga all'annullatore
\end{prop}
\begin{prop} Alcune propriet\'a dell'annullatore e del luogo di zeri
\begin{itemize}
\item[(i)] $ S \subseteq T \implica Ann(T)\subseteq Ann(S) $
\item[(ii)] $ \forall f \in V^* \quad Ann (f)=\phi(Ker f ) $ con $\phi $ isomorfismo canonico $V \to  \left( V^* \right) ^* $
\item[(iii)] $\forall U $ sottospazio di V $\quad Ann(Ann(U))=\phi(U) $
\item[(iv)] $Ann(Ann(W))=W$
\item[(v)] $Z(Ann(W))=W $
\end{itemize}
\proof \bianco \begin{itemize}
\item[(i)]
$$f \in Ann(T) \implica f(v)=0 \quad \forall v \in T \implica f(v)=0 \quad \forall v\in S \implica f \in Ann(S)$$
\item[(ii)]
$$ Ann(f)= \{ h \in  \left( V^* \right) ^* \, \vert \, h(f)=0 \} =
\{ \phi (x) \in  \left( V^* \right) ^* \, \vert \, \phi(x)(f)=f(x)=0 \}=$$
$$ = \phi \left( \big\{  x \in V \, \vert \, f(x)=0\big\}\right)=\phi(Ker f ) $$
\item[(iii)]
Sia $\dim V = n $ e $ U $ sottospazio di $V$  allora
$$ \dim Ann(U)=n -\dim U $$
$$ \dim Ann(Ann(U))=n -\dim Ann(U)=n -n + \dim U $$
Ma poich\`e $ U \cong \phi(U) $ vale che $$\dim \phi (U) = \dim U = \dim (Ann (Ann(U))$$
Quindi poich\`e i due sottospazi hanno la stessa dimensione basta dimostrare una sola inclusione
$$ \forall u \in U \quad \forall \psi \in Ann(U) \quad \phi (u)(\psi)=\psi(u)=0$$
quindi vale che $\phi(U)\subseteq Ann(Ann(U))$
\end{itemize}
\endproof
\end{prop}
$$ 
\begin{tikzcd} 
G_k(V)
\arrow[r,"Ann"]
&G_{k-n}(V^*) 
\arrow[d,"Z"]
\arrow[r,"Ann"]
& G_k\left( \left( V ^*\right) ^* \right) 
\\
& G_k(V) 
\arrow{ur} [leftrightarrow]{\phi} 
\end{tikzcd}
$$
Il diagramma commuta dove $\phi$ \`e ottenuto dall'isomorfismo canonico tra uno spazio ed il suo bi-duale
\newpage
\subsection{Trasposta}
\begin{defn}[Applicazione trasposta]\bianco
Sia $f:\, V \to W $ allora definiamo l'applicazione trasposta
$${}^t f:\, W^*\to V^* \qquad  {}^t f(\psi)=\psi \circ f  $$
\end{defn}
\begin{oss}Osserviamo che la funzione \`e ben definita.\\
Supponiamo che $\psi \in W^*$ allora $\psi:\, W \to \K $ quindi
$$\begin{tikzcd}
V
\arrow[r,"f"]
&W 
\arrow{r}{\psi}
&\K
\end{tikzcd}$$
ovvero $\psi \circ f \in V^* $ quindi ${}^tf:\, W^* \to V^*$
\end{oss}
\begin{prop}La trasposta \`e lineare\\
Siano $\psi , \varphi \in W^* $ allora
$$ {}^t f( \psi + \varphi ) = (\psi + \varphi ) \circ f = \psi \circ f + \varphi \circ f =  {}^t f( \psi ) +  {}^t f( \varphi)  $$
Sia $a \in \K $ 
$$  {}^t f(a \psi ) =(a\psi ) \circ f = a(\psi \circ f ) =  a \cdot {}^t f( \psi )$$
\endproof
\end{prop}
\begin{prop}Per l'applicazione trasposta sono veri i seguenti fatti
\begin{itemize}
\item[(i)] ${}^t \left( {}^t (f) \right)=f$
\item[(ii)] Se $h:\, W \to Z$ lineare allora
$$ {}^t(h \circ f ) = {}^t f \circ {}^t h$$
\item[(iii)]$ Ker ({}^t f )=Ann(Im (f) ) $ 
\item[(iv)] $Imm ({}^t f )=Ann(Ker (f) ) $
\item[(v)] Se $\B $ base di V e $\D$ base di W allora
$$ M_{\B^*} ^{\D^*} \left({}^tf \right) = {}^t \left( \AB (f) \right)$$
\end{itemize}
\proof \bianco
\begin{itemize}
\item[(i)]
$$
\begin{tikzcd}
V 
\arrow[r,"f"]
\arrow{d}{\phi_V} 
&W 
\arrow{d}{\phi_W}\\
\left( V ^* \right)^* 
\arrow{r}{{}^t \left( {}^t f\right)}
&\left( W ^* \right)^* 
\end{tikzcd} $$
Devo dimostrare che il diagramma commuta quindi che 
$$ \phi_W \circ f = {}^t \left( {}^t f\right) \circ \phi_V \text{ ovvero }$$
$ \forall v \in V $
$$( \phi_W \circ f)(v) =  \left( {}^t \left( {}^t f\right) \circ \phi_V \right) (v) \text{ ovvero } $$
$\forall \psi \in W^* $ 
$$( \phi_W \circ f)(v) (\psi) =  \left( {}^t \left( {}^t f\right) \circ \phi_V \right) (v) (\psi)  $$
Mostriamo che \`e vera l'ultima uguaglianza 
$$ ( \phi_W \circ f)(v) (\psi) =( \phi_W (f(v)) ) (\psi)=\psi(f(v))=(\psi \circ f ) (v) $$
 $$\left( {}^t \left( {}^t f\right) \circ \phi_V \right) (v) (\psi)= 
{}^t \left( {}^t f \right) (  \phi_V(v))( \psi) =\left( {}^t \left( {}^t f \right)  \circ \phi_V (v)\right)(\psi)=$$
$$= \left( \phi_V \circ {}^t f \right) (\psi)= \phi_V(v)\left( {}^t f \circ \psi \right)=\left( {}^t f \circ \psi \right) (v)=(\psi \circ f ) (v) $$
\item[(ii)] ${}^t ( h \circ f ) :\, Z^* \to V^* $ quindi $\forall \psi \in Z^*$
$${}^t ( h \circ f )(\psi) = \psi \circ h \circ f = \left( {}^t h ( \psi) \right) \circ f ={}^t f \left( \left( {}^t h ( \psi) \right) \right)=\left( {}^t f \circ {}^t h \right) (\psi)  $$
\item[(iii)]Mostriamo entrambe le inclusioni\\
$\subseteq \quad \forall \psi \in Ker \left( {}^t f \right)$ vale 
$$ \left( {}^t f \right) (\psi) = 0 \, \implica \, (\psi \circ f ) = 0 \implica \forall v \in V  \, \psi(f(v))=0 \implica \psi \in Ann (Imm(f)) $$
$\supseteq \quad \forall \psi \in Ann(Imm(f))$ vale $ \forall v \in V$
$$ \psi(f(v))=0 \implica ( \psi \circ f ) (v)=0 \implica {}^t f ( \psi)=0 \implica \psi \in Ker \left( {}^t f \right) $$
\item[(iv)] Dalla $(iii) $ 
$$ Ker \left( {}^t \left( {}^t f \right) \right)= Ann \left( Imm \left( {}^t f \right) \right) $$
Applicando l'annullatore
$$ Im\left( {}^t f \right) = Ann \left( Ann \left( Imm \left( {}^t f \right) \right)\right)=Ker \left( {}^t \left( {}^t f \right) \right)=Ker (f)$$
\end{itemize}
\end{prop}
\begin{cor} Rango della trasposta
\proof
$$ rk\left( {}^t A\right) = \dim Im \left( {}^t A \right)=\dim Ann (Ker (f)) = n -\dim Ker(f)=rk(A)$$
\endproof
\end{cor}
%\end{document}

