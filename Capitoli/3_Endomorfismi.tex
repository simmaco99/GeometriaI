%%\documentclass[a4paper,12pt]{article}
\usepackage[a4paper, top=2cm,bottom=2cm,right=2cm,left=2cm]{geometry}

\usepackage{bm,xcolor,mathdots,latexsym,amsfonts,amsthm,amsmath,
					mathrsfs,graphicx,cancel,tikz-cd,hyperref,booktabs,caption,amssymb}
\hypersetup{colorlinks=true,linkcolor=blue}
\usepackage[italian]{babel}
\usepackage[T1]{fontenc}
\usepackage[utf8]{inputenc}


\newcommand{\N}{\mathbb{N}} %naturali 
\newcommand{\R}{\mathbb{R}} %Reali
\newcommand{\K}{\mathbb{K}} % Campo K 
\newcommand{\C}{\mathbb{C}}
\newcommand{\F}{\mathbb{F}}

\newcommand{\B}{\mathfrak{B}} %Base B
\newcommand{\D}{\mathfrak{D}}%Base D
\newcommand{\RR}{\mathfrak{R}}%Base R 
\newcommand{\Can}{\mathfrak{C}}%Base canonica
\newcommand{\Rif}{\mathfrak{R}}%Riferimento affine
\newcommand{\AB}{M_\D ^\B }% matrice applicazione rispetto alla base B e D 

\newcommand{\vett}{\overrightarrow}
\newcommand{\sd}{\sim_{SD}}%relazione sx dx
\newcommand{\sbarra}{\backslash} %% \ 
\newcommand{\ds}{\displaystyle} 
\newcommand{\alla}{^}  
\newcommand{\implica}{\Rightarrow}
\newcommand{\ses}{\Leftrightarrow} %se e solo se
\newcommand{\tc}{\quad \text{ t. c .} \quad } % tale che 
\newcommand{\spazio}{\vspace{0.5 cm}}
\newcommand{\bbianco}{\textcolor{white}{,}}
\newcommand{\bianco}{\textcolor{white}{,} \\}% per andare a capo dopo 																					definizioni teoremi ...
\newcommand{\nvett}{v_1, \, \dots , \, v_n} % v1 ... vn
\newcommand{\ncomb}{a_1 v_1 + \dots + a_n v_n} %a1 v1 + ... +an vn
\newcommand{\nrif}{P_1, \cdots , P_n} 
\newcommand{\bidu}{\left( V^\star \right)^\star}

%per creare teoremi, dimostrazioni ... 
\theoremstyle{plain}
\newtheorem{thm}{Teorema}[section] 
\newtheorem{ese}[thm]{Esempio} 
\newtheorem{ex}[thm]{Esercizio} 

\newtheorem{cor}[thm]{Corollario} 
\newtheorem{lem}[thm]{Lemma} 
\newtheorem{al}[thm]{Algoritmo}
\newtheorem{prop}[thm]{Proposizione} 
\theoremstyle{definition} 
\newtheorem{defn}{Definizione}[section] 
\theoremstyle{remark} 
\newtheorem{oss}{Osservazione} 


% diagrammi commutativi tikzcd
% per leggere la documentazione texdoc

%\begin{document}
\section{Somma diretta multipla}
\begin{prop} \label{sommamultipla}Sia $V$ uno spazio vettoriale e siano $W_1 ,\cdots, W_k$ sottospazi vettoriali.\\
I seguenti fatti sono equivalenti
\begin{enumerate}
\item $\forall v \in W_1 + \cdots + W_k $
$$\exists \, ! \,  v_1, \cdots , v_k \text{ con } v_j \in W_j \quad v=v_1+\cdots+v_k$$
\item 
$$ v_1+ \cdots + v_k =0 \wedge v_j \in W_j \implica v_1=\cdots=v_k=0 $$
\item Se $\B_j$ \'e base di $W_j$ allora $$ \B=\B_1\cup \cdots \cup \B_k \text{ \'e base di  } W_1+ \cdots + W_k$$
\item $$\dim ( W_1 + \cdots + W_k ) =\dim W_1 + \cdots + \dim W_k $$
\end{enumerate}
La dimostrazione \'e lasciata come esercizio
\end{prop}
\begin{defn}[Somma diretta multipla]\bianco
Se si verificano questi fatti $W_1, \cdots , W_k $ sono in somma diretta e si scrive 
$$ W_1 \oplus \cdots \oplus W_k $$
\end{defn}
\spazio  
\newpage

\section{Alcune nozioni sugli endomorfismi }
Ricordiamo cosa \'e un endomorfismo
\begin{defn}[Endomorfismo]\bianco
Sia $V$ uno spazio vettoriale allora definiamo endomorfismo una funzione
$ f: \, V \to V $ lineare\\
Denotiamo inoltre con 
$$ End(V) = \{ f :\, V \to V \, \vert \, \text{ endomorfismo } \} $$  
\end{defn}
\begin{prop}Se $\dim V= n $.
Allora
$$ \dim (End(V))=n^2 $$
\proof
Fissata una base $\B=\{ \nvett \} $ di V, per definire un endomorfismo basta assegnare i vettori su una base.\\
Ogni vettore della base $\B$ pu\'o essere mandato in un qualsiasi altro vettore della base.\\Per ogni vettore ho $n$ scelte, i vettori sono $n$ da cui $n^2 $
\endproof
\end{prop}

\subsection{Alcune definizioni}

\begin{defn}[Autovalore e autovettore]\bianco
Sia $f \in End(V) $ \\
$ \lambda \in \K $ si dice autovalore per $f$ se 
$$ \exists v \in V \quad v\neq 0  \tc f(v)= \lambda v $$
In tal caso $v$ \'e detto autovettore per $\lambda$
\end{defn}
\begin{defn}[Spettro]
$$Sp(f)=\{ \lambda \in \K \, \vert  \lambda \text{ autovalore per } f \} = \{ \lambda \in \K \, \vert \, \ker(f-\lambda id) \neq 0 \}$$ 
\end{defn}
\begin{defn}[Autospazio e molteplicit\'a geometrica] Sia $\lambda $ autovalore per $f$
$$ V_\lambda =\{ v \in V \, \vert \, f(v)=\lambda v \}  = \ker (f-\lambda id) $$
L'autospazio $V_\lambda$ \'e formato dagli autovettori per $f$ \\
La dimensione dell'autospazio $V_\lambda$ \'e detta  molteplicit\'a geometrica di $\lambda$
$$ m_g (\lambda)=\dim V_\lambda $$
\end{defn}
\spazio
\begin{defn}[Polinomio caratteristico]\bianco
Sia $A\in M(n , \K) $ 
$$ P_A(t)= \det (A -tI) \in \K[t]$$
Possiamo definire il polinomio anche sugli endomorfismi 
$$ P_f(t)=P_A(t) \quad \text{ dove } A = M_\B(f) $$
con $\B$ base arbitraria di $V$
\end{defn}
\begin{defn}[Molteplicit\'a algebrica]
$\forall \lambda \in Sp(f)$ $$m_a(\lambda )  \text{ \'e la molteplicit\'a algebrica di }  \lambda  \text{ come radice del polinomio caratteristico} $$
\end{defn}
\newpage
\subsection{Alcune propiet\'a}
\begin{prop}Gli autospazi sono in somma diretta multipla.
Sia $$\{ \lambda_1,\cdots, \lambda_s \} \subseteq Sp(f) \text{ con }\lambda_i \neq \lambda_j  \text{ se }  i\neq j$$
Allora $$ V_{\lambda_1} \oplus \cdots \oplus V_{\lambda_s} $$
\proof Per induzione su $s\geq 1 $ utilizzando  la definizione 2 della somma diretta multipla (\ref{sommamultipla}) \
Il passo base \'e banale \\
Sia 
\begin{equation} \label{equ1}
v_1 + \cdots + v_s=0 
\end{equation} 

applicando $f$  alla (\ref{equ1}) otteniamo 
\begin{equation}\label{equ2}
 \lambda_1 v_1 + \cdots + \lambda_s v_s=0
\end{equation}
 moltiplicando (\ref{equ1}) per $\lambda_k$ otteniamo 
 \begin{equation} \label{equ3}
\lambda_k v_1 + \cdots + \lambda_k v_s=0 
 \end{equation}
Sottraendo (\ref{equ2}) - (\ref{equ3})  e raccogliendo otteniamo 
$$ ( \lambda_1 -\lambda_k) v_1 + \cdots + ( \lambda_{k-1} - \lambda_k) v_{k-1} =0$$
Quindi poich\'e gli autovettori sono distinti, applicando l'ipotesi induttiva vale
$v_i=0 \quad \forall i $
\end{prop}

\spazio

\begin{prop}\label{molt_alg_e_geo}
$ \forall \lambda \in Sp(f) $
$$ 1\leq m_g(\lambda) \leq m_a(\lambda) \leq n=dim(V) $$
\proof  Poniamo $m_g(\lambda) = g$
$$\D=\{ v_1 , \, \cdots , \, v_{g} \} \text{ una base di } V_\lambda $$
Estendiamo, tale base a $\B$  base di $V$ 
$$ A= M_\B (f)= 
\left (
\begin{array}{c|c}
\lambda I_{g}  & M \\
\hline
0 & N \\
\end{array}
\right )$$
$$P_f(t)=P_A(t)=\det  \left (
\begin{array}{c|c}
(\lambda -t ) I_{d_\lambda}  & M \\
\hline
0 & D -t  I  \\
\end{array}
\right )=( \lambda -t )^{g} P_D(t) $$
da cui segue che $m_a(\lambda) \geq  g$
\endproof
\end{prop}
\newpage
\subsection{Ideali di un endomorfismo}
\begin{defn}[Valutazione polinomio su endomorfismo]\bianco
Sia $V $ uno spazio vettoriale su $\K$ e sia $f\in End(V) $
$$ p(t)=a_0 t^0 + \cdots + a_k t^k \in \K[t]$$
Allora definiamo $p(f)\in End(V)$ come 
$$ p(f)=a_0 f^0 + \cdots+ a_k f^k $$
dove indichiamo $f^0=id_V $ e $f^i= \underbrace{ f \circ \cdots \circ  f}_{\text{ i volte } } $
\end{defn}
\spazio
Fissato $f\in End(V) $ 
$$ \phi: \, \K[t]\to End(V) \quad \phi(p(t))=p(f)$$ 
Ovvero valuta ogni polinomio in $f$ 
\begin{ex} Dimostrare che $\phi $ \'e omomorfismo di anelli 
\end{ex}
\begin{oss}\label{commu}Poich\'e $\K[t]$ un anello commutativo e $\phi$ \'e omomorfismo di anelli
$$ p_1(f)\circ p_2(f)=p_2(f)\circ p_1(f)$$
\end{oss}
\spazio
\begin{defn}[Ideale di un endomorfismo]\bianco
Sia $f\in End(V) $ e sia $\phi$ come sopra  allora definiamo l'ideale di $f$ come 
$$ I(f)=\ker \phi =\{  p(t) \in \K[t]\, \vert \, p(f)=0\in End(V)\}$$ 
\end{defn}
 \spazio
 \begin{lem}[Gli ideali sono non banali]
$$\forall f \in End(V) \, I (f)\neq \{ 0 \} $$
\proof
Poich\'e $\dim End(V)= n^2$, fissato $m>n^2 $ ne segue che 
$$ f^0 , \, f^1, \, \cdots \, , \, f^m $$
 non sono linearmente indipendenti (sono di pi\'u della dimensione) da cui
$$ \exists a_0 f^0 +\cdots + a_m f^m = 0 \in End(V) \quad \text{ tali che } \exists a_j \neq 0 $$ 
Dunque
$$ p(t)= a_0 +a_1 t + \cdots + a_m t^m \in I(f) $$ 
inoltre visto che esiste almeno un coefficiente non nullo, $p(t)\neq 0 $ ovvero $I(f) \neq \{ 0 \}$
\endproof
\end{lem}
\spazio
\subsubsection{Teorema di Hamilton-Cayley}

\begin{lem}
Sia $f\in End(V)$ con polinomio caratteristico completamente fattorizzabile.\\
Allora $$p_f(t) \in I (f)$$
\proof
Sia 
$$ p_f(t)=( t-\mu_1) \cdots ( t-\mu_n)$$
il polinomio caratteristico di $f$.\\
Per vedere che il polinomio valuto in $f$ sia il polinomio nullo, basta osservare che esso annulla una base.\\
Prendiamo una base$\B=\{ \nvett \}$ tale che la $M_\B(f) $ sia triangolare superiore (il motivo per cui tale base esiste viene dimostrato successivamente, quando parleremo di endomorfismi triangolabili) quindi
$$ f(v_1)=\mu_1 v_1 $$
$$ f(v_2)=\mu_2 v_2 + \star v_1 $$
Calcoliamo $p_f(f)$ su $v_1$
$$ p_f(f)(v_1)=( f-\mu_1 Id ) \circ \cdots \circ  ( f-\mu_n Id )(v_1)\bm{= } ( f-\mu_2  Id ) \circ \cdots \circ  ( f-\mu_n Id )\circ (f-\mu_1 Id)(v_1)=0 $$
Dove il secondo uguale viene giustificato dall'osservazione~\ref{commu}\\
Calcoliamo su $v_2$
$$ p_f(f)(v_2)=( f-\mu_1 Id ) \circ \cdots \circ  ( f-\mu_n Id )(v_2)=  ( f-\mu_3  Id ) \circ \cdots \circ  ( f-\mu_n Id )\circ (f-\mu_1 Id)\circ (f-\mu_2 Id) (v_2)= $$$$( f-\mu_3  Id ) \circ \cdots \circ  ( f-\mu_n Id )\circ (f-\mu_1 Id) (\star v_1 )=0  $$
Per induzione si dimostra che annulla una base e dunque vale la tesi \endproof
\end{lem}

Mostriamo ora la generalizzazione del lemma precedente
\begin{thm}[Hamilton-Cayley]\bianco
$\forall f \in End(V)$
$$ p_f(t) \in I(f) $$
\proof Possiamo considerare $\mathbb{F}$ campo di spezzamento del polinomio $p_f(t) $ in questo caso 
$$
\begin{tikzcd} 
V \arrow[r, "f"] \arrow[d,""]
& V  \arrow[d, ]\\ 
\K ^ n \arrow[r, "A"]  \arrow[d,hook,"i"] 
& \K ^ m  \arrow[d,hook,"i"]  \\
\mathbb{F} \arrow{r}{A_\mathbb{F}} 
&\mathbb{F}
\end{tikzcd}
$$
Dove con $A_\mathbb{F} = A \in (M,n \mathbb{F }) $.\\
Ora poich\'e
 $\F $ \'e campo di spezzamento del polinomio caratteristico vale che $ P_A(A)=0\in M(n,\F) $ per il lemma precedente.\\
Ora l'uguaglianza precedente vale anche in $\K$ quindi vale la tesi
\endproof
\end{thm}
\newpage
\begin{lem} Sia $p(t) \in I (f)$
$$ \lambda \in Sp(f) \implica p(\lambda )=0$$
\proof Essendo $\lambda$ un autovalore 
$$ \exists v \in V \, v\neq 0 \tc f(v)=\lambda v $$
Inoltre sappiamo $p(f)(v)=0 $ infatti $p(t) $ appartiene all'ideale di $f$ \\
Inoltre $$ p(f)(v)= \left( a_0 I + a_1 f + \cdots + a_k f^k\right) (v)=a_0 v + \lambda a_1 v + \cdots + \lambda^k a_k v =$$ 
$$ = v( a_0+ a_1 \lambda + \cdots + a_k \lambda ^k )  $$
Ora l'ultima espressione vale $0$ inoltre essendo $v$ autovettore $v\neq 0 $ da cui il termine nella parentesi deve essere uguale a 0.\\
Ma il termine nella parentesi non \'e altro che $ p(\lambda ) $
\endproof
\end{lem}
\subsubsection{Polinomio minimo}
Essendo $\K[t]$ un PID, tutti i suoi ideali sono mono generati

\begin{defn}[Polinomio minimo di $f$ ]\bianco
Sia $q_t$ il polinomio monico che genera $I(f)$
\end{defn}
\spazio

\begin{oss} Possiamo applicare il lemma precedente al polinomio minimo.\\
Sia  $\lambda \in Sp(f) $ allora  
$$ P_f(t)=\pm ( t-\lambda)^{m_\lambda} q(t)$$
dunque 
$$q_t(f)=\pm ( t-\lambda) ^{r_\lambda } q_1(t) \quad 1 \leq r_\lambda \leq m_\lambda $$
\end{oss}
\newpage
\subsubsection{Polinomio minimo di un vettore}
Sia $v\in V $ e $f\in End(V) $ allora definiamo la valutazione in $f(v) $ come 
$$ \K[t] \to End(V) \to V \qquad p \to p(f) \to p(f)(v)$$
\spazio
Come per la valutazione su un endomorfismo possiamo considerare $I(f,v) $ e considerare il polinomio minimo $\mu_{f,v} $ come il generatore monico dell'ideale
\begin{lem}$$\mu_{f,v} \, \vert \, \mu_f$$
\proof
La divisibilit\'a deriva dal fatto che $ I(f)  \subseteq I(f,v)$.\\
Infatti$$\forall p \in \K[t] \quad p(f) \equiv 0 \quad \implica \quad  p(f)(v)=0$$
\endproof
\end{lem}
\begin{prop} \label{A}Sia $ \nvett $ un insieme di generatori di $V$.\\
Allora
$$ \mu_f = m.c.m \left( \mu_{f,v_1} , \cdots, \mu_{f,v_n} \right) =m \text{ monico } $$
\proof Visto che $ \mu_{f,v_i} \, \vert \mu_f $ allora $m \, \vert \, \mu_f$.\\
Ora $ m \in I(f)$ infatti  
$$\forall i \quad \exists h_i \in \K[t] 
\quad m = h_i \mu_{f,v_i} \quad \text{ e}  \quad \forall v =\sum_{i=1}^n a_i v_i $$
$$m(f)(v)=m(f) \left( \sum_{i=1}^n a_i v_1 \right)= \sum_{i=1}^n a_i \cdot m(f) (v_i) = \sum_{i=1}^n a_1 \left(h_i \cdot \mu_{f,v_1} \right) (f(v_i))= $$
$$=\sum_{i=1}^n a_1 \cdot ( h_i(f) \, \circ \, \mu_{f,v_i} )(v_1)=\sum_{i=1}^n a_1 \cdot ( h_i(f) ( \, \mu_{f,v_i} (v_1))=0 $$
quindi $\mu_f \, \vert m $.\\
Poich\'e valgono  entrambe le divisibilit\'a e poich\'e entrambi sono monici, vale la tesi.
\endproof
\end{prop}
\newpage
\subsubsection{Calcolare il polinomio minimo}
\paragraph{ Primo metodo}
\begin{oss} Se il grado del polinomio minimo di $f$ \'e $d$ allora
$$ id, f,f^2 ,\cdots , f^{d-1} \text{ sono linearmente indipendenti } $$
Supponiamo che non siano indipendenti allora
$$ a_0 \cdot i_d +a_1 f + \cdots  + a_{d-1} f^{d-1} =0 \quad \exists a_i \neq 0$$
da cui segue che il polinomio $p(t)=a_0 + a_1 t + \cdots a_{d-1}f^{d-1} \in I(f) $ ma ci\'o \'e assurdo perch\'e $p(t) \neq 0 $ poich\'e $a_i \neq 0 $ ed ha grado minore del polinomio minimo
\end{oss}
Dunque per trovare il polinomio minimo, analizzo le prime potenze di $f$ finch\'e non trovo la prima lista di potenze non linearmente indipendenti.\\
Se noto che $ id, f, \cdots, f^{d-1} $ sono linearmente indipendenti ma $ id, f, \cdots, f^{d-1} ,f^d $ non lo \'e allora
$$ f^d= a_0 id + a_1 f + \cdots + a_{d-1} f^{d-1} $$
da cui il polinomio minimo di f \'e 
$$ t^d + a_{d-1} t^{d-1} + \cdots + a_0 $$
\begin{ex}
\end{ex}Sia $f \in End(V) $ uno spazio vettoriale reale e sia $\B$ una base di $V$ tale che 
$$A= M_\B(f) = \begin{pmatrix} 0  & 1 & 1 \\
											0  & 1 & 0 \\
											-1 & 1 & 2  \end{pmatrix}$$
Notiamo che $A $ e $I_3$ sono linearmente indipendenti quindi grado del polinomio minimo $\geq 2 $
$$ A^2 = \begin{pmatrix} -1 & 2 & 2 \\
										0 & 1 & 0 \\
										-2 & 2 & 3  \end{pmatrix} $$
$I_3,A $ e $A^2 $ sono dipendenti? $\exists \alpha, \beta \in \R $ tale che $A^2 =\alpha I_3 + \beta A $ ? 
$$ \begin{pmatrix}
-1=\beta & 2 = \alpha & 2 =\alpha \\
0=0		& 	 1= \alpha+\beta & 0=0 \\
-2=-\alpha & 2=\alpha & 3 =2\alpha + \beta 
\end{pmatrix}$$
$$ A^2 = 2 A - I \quad \implica \quad A^2 -2A +I $$
quindi il polinomio minimo di $f$ \'e 
$$ t^2 -2+1 $$
\paragraph{Secondo metodo}
So che il polinomio minimo, divide un qualsiasi elemento dell'ideale, in particolare posso prendere il polinomio caratteristico.\\
Se uso il polinomio caratteristico, so che hanno gli stessi fattori irriducibili
\newpage
In questo capitolo il polinomio minimo verr\'a indicato con la lettera $\mu$ e non $q$
\paragraph{Terzo metodo}
Questo metodo sfrutta il polinomio minimo di un vettore
\begin{ex}
$$ A = \begin{pmatrix}	1 & 2 & 3 \\
									0 & 1 & 0 \\
									0 & 0 & 2 
\end{pmatrix} \in M(3,\R)$$
Prendo $\B= \{ e_1, e_2 , e_3\} $ base di $M(3,\R)$ e calcolo i 3 polinomi minimi $\mu_{A,e_i}$
$$ A e_1 = e_1 \quad \implica \quad \mu_{f,e_1}(t) = t-1 $$
$$
\begin{tikzcd} e_2 \arrow[r,"A"] &\begin{pmatrix}
2 \\ 1 \\0 
\end{pmatrix} \arrow[r,"A"] &\begin{pmatrix}
4 \\ 1 \\0
\end{pmatrix} = 2\begin{pmatrix}
2 \\ 1 \\ 0
\end{pmatrix} - \begin{pmatrix}
0 \\ 1 \\0 
\end{pmatrix} 
\end{tikzcd} \quad \implica \quad \mu_{f,e_2} = (t-1)^2$$

$$
\begin{tikzcd} e_3 \arrow[r,"A"] &\begin{pmatrix}
3 \\ 0 \\2 
\end{pmatrix} \arrow[r,"A"] &\begin{pmatrix}
9 \\ 0 \\4
\end{pmatrix} = 3\begin{pmatrix}
3 \\ 0 \\2 
\end{pmatrix} - 2\begin{pmatrix} 0 \\ 0 \\1
\end{pmatrix} 
\end{tikzcd} \quad \implica \quad \mu_{f,e_3} = (t-1)(t-2)$$
Quindi per la proposizione \ref{A} vale
$$ \mu_f = (t-1)^2 (t-2)$$
\end{ex}
\newpage
\subsection{Endomorfismi diagonalizzabili}
Prima della prossima proposizione ricordiamo una definizione data nel primo capitolo
\begin{defn} Una matrice $A$ si dice diagonale se
$$ \forall i\neq j \quad [A]_{ij}=0 $$
\end{defn}
\spazio

\begin{prop} Sia $f \in End(V)$.\\
I seguenti fatti sono equivalenti
\begin{itemize}
\item[(i)] $\exists \B$ base di $V$ fatta di autovettori  di $f$
\item[(ii)]$\exists \B$ base di $V$ tale che $M_\B(f)$ \'e diagonale
\item[(iii)]$$V= \bigoplus_{\lambda \in Sp(f)} V_\lambda$$
\item[(iv)] Il polinomio caratteristico ha tutte le radici in $\K$ e
$$ \forall \lambda \in Sp(f) \quad  m_a(\lambda)=m_g(\lambda)$$
\item[(v)]Il polinomio minimo ha tutte le radici in $\K$ di molteplicit\'a $1$

\end{itemize}
\proof \bianco
\begin{itemize}
\item(i) $\ses $(ii)\\ 
La dimostrazione dell'equivalenza \'e immediata
\item (i) $\implica $(iii)\\
Per ipotesi  $\exists \B$ base di autovettori.\\
Suddivido $$\B=\B_1 \cup \dots \cup \B_k \text{ con }\B_i \subseteq V_{\lambda_i} $$ da cui
$$ V= Span(\B_i) \oplus \dots \oplus Span(\B_k) \subseteq V_{\lambda_1} \oplus \dots \oplus V_{\lambda_k}$$
Poich\'e l'altra inclusione \'e sempre vera, ho la tesi
 \item(iii) $\implica $(i)\\
  Se $\B_i$ \'e base di $ V_{\lambda_i} $ allora 
 $$\B_{\lambda_1} \cup \dots \cup B_{\lambda_k} \text{ \'e base di  } V $$
infatti, segue dalla definizione di somma diretta multipla, inoltre se $ v \in \B_{\lambda_i}$ allora $v\in V_{\lambda_i}$ dunque \'e un autovettore.\\
Per quanto detto sopra $\B$ \'e una base di $V$ composta da autovettori per $f$
\item(ii) $\implica $(iv)\\
  $\exists \B$ base di autovettori di $f$ tale che 
$$A= M_\B(f)= \begin{bmatrix}\lambda_1 I_{d_{1}}  & & \\ & \ddots & \\ & &  \lambda_k I_{d_{k}}
\end{bmatrix}$$ 
con $d_{1} + \dots + d_{k}=\dim V=n $\\
Allora 
$$P_f(t) = \det (A-tI)=(\lambda_1 -t ) ^{d_{1}} \cdots (\lambda_k -t ) ^{d_{k}}$$
Da questo segue che $P_f(t)$ \'e completamente fattorizzabile.\\
Poich\'e $\B$ contiene $d_{i} $ vettori relativi a $\lambda_i$ allora $$m_g(\lambda_i) \geq d_{i}$$
Inoltre, per la proposizione~\ref{molt_alg_e_geo}
 $$d_i=m_a(\lambda_i) \geq m_g({\lambda_i})$$
 Poich\'e valgono entrambe le disuguaglianze, otteniamo l'uguaglianza voluta
 
\item(iv) $\implica $(iii)\\
 So che 
$$ V_{\lambda_1} \oplus \dots \oplus V_{\lambda_k} \subseteq V $$
ma 
$$ \dim \left( V_{\lambda_1} \oplus \dots \oplus V_{\lambda_k}  \right) = \dim V_{\lambda_1} + \dots + V_{\lambda_k}=\sum_{i=1}^k d_{\lambda_1} =\sum_{i=1} ^k m_{\lambda_i} =n $$
Poich\'e vale una disuguaglianza e hanno la stessa dimensione, vale l' uguaglianza
\end{itemize}
Le altre implicazioni sono lasciate come esercizio
\endproof
\end{prop}
\spazio
\begin{defn}[Diagonalizzabile] \bianco
$f\in End(V) $ si dice diagonalizzabile se verifica una delle propriet\'a sopra elencate.
\end{defn}


\newpage
\subsubsection{Simultanea diagonalizzabilit\'a }
\begin{defn}[Simultaneamente diagonalizzabili]\bianco
$f,g\in End(V) $ si dicono simultaneamente diagonalizzabili se ammettono una base comune di autovettori
\end{defn}
\spazio
\begin{lem}
$f,g \in End(V) $ 
$$ f \circ g = g \circ f \implica \forall \lambda\in Sp(f) \quad V_\lambda(f) \text{ \'e } g\text{-invariante } $$
\proof
Sia $\lambda \in Sp(f) $\\
$$\forall v \in V_\lambda(f)=\ker ( f - \lambda Id) $$
$$f(g(v))=g(f(v))=g(\lambda v ) = \lambda g(v) \quad \implica \quad g(v) \in V_\lambda(f)$$
\endproof
\end{lem}
\spazio
\begin{prop} $f,g \in End(V)$  diagonalizzabili
$$ f,g \text{ simultaneamente diagonalizzabili } \ses g\circ f = f\circ g $$
\proof $\implica$ 
La possiamo fare in 2 modi 
\begin{itemize}
\item[(i)] $\exists \B=\{ \nvett \} $ base di $V$ con autovettori di $f$ e $g$\\
$$A=  M_\B(f) = \left( \begin{array}{cc} \begin{array}{cc} \lambda_1 &  \\ & \lambda_2 \end{array} & \star \\  0 &  \begin{array}{cc} \ddots &  \\ & \lambda_n  \end{array} \end{array}\right) \quad 
B=M_\B(f) = \left( \begin{array}{cc} \begin{array}{cc} \mu_1 &  \\ & \mu_2 \end{array} & \star \\  0 &  \begin{array}{cc} \ddots &  \\ & \mu_n \end{array} \end{array}\right)
$$
con  $ \lambda_i \in Sp(f) $ e $\mu_i \in Sp(g) $

$$ M_\B(f\circ g ) =M_\B (g\circ f ) = \left( \begin{array}{cc} \begin{array}{cc} \lambda_1 \mu_1 &  \\ & \lambda_2 \mu_2 \end{array} & \star \\  0 &  \begin{array}{cc} \ddots &  \\ & \lambda_n \mu_n \end{array} \end{array}\right)
$$

\item[(ii)] Sia $ \B $ come sopra.\\
$ f\circ g = g \circ f $ se sono uguali su una base quindi se
$$ \forall v_i \in \B \quad f(g(v_i))=g(f(v_1))$$
$$ f(g(v_i))=f(\mu_i v_i) =\mu_i f(v_i)=\mu_i \lambda_i v_1 $$
$$ g(f(v_i))=g(\lambda_i v_i) =\lambda_i g(v_i)=\mu_i \lambda_i v_1 $$
\end{itemize}

Mostriamo ora la freccia $\Leftarrow$
$$Sp(g)=\{ \lambda_1, \, \dots , \, \lambda_k \}$$ 
$$ f \text{ diagonalizzabile } \quad \ses \quad V = V_{\lambda_1 } (f) \oplus \dots \oplus V_{\lambda_k} (f) $$
Ora possiamo considerare la restrizione di $g$ su questi autospazi che per il lemma precedente sono $g$-invarianti.\\
$$ g\text{ diagonalizzabile } \implica g_{\vert V_{\lambda_i}} \text{ diagonalizzabile } \implica \exists \B_i {base di } V_{\lambda_i} \text{ di autovettori di } g $$
Ora la base $\B_i$ contiene sia autovettori di $f$ (gli elementi dell'autospazio sono autovettori) che di g, dunque la base cercata \'e $\{ \B , \dots , \B_k \} $
\endproof
\end{prop}
\spazio
\newpage

\subsection{Endomorfismi triangolabili }

\begin{defn}[Bandiera indotta da una base] \bianco
Sia $\B=\{ \nvett \} $ una base di V.\\$\B$ induce una bandiera di sottospazi
$$ Span(v_1) \subseteq Span(v_1, v_2) \subseteq \dots \subseteq Span \B =V $$
Indicata con $\mathfrak{F}_\B$.\\
Per comodit\'a nella trattazione, quando \'e presente una base, indicheremo con
$$ V_1 = Span(v_1)$$
$$ V_k=Span(v_1, \dots,  v_k) $$
\end{defn}
\spazio
\begin{defn}[Sottospazio $f$-invariante]
Sia $f\in End(V)$ e sia $W \subseteq V $ un sottospazio.\\
$W$ \'e un sottospazio $f$-invariante se 
$$ f(W)\subseteq W $$
\end{defn}
\begin{defn}[Bandiera invariante]\bianco
Si dice che $\mathfrak{F}_\B$ \'e $f$-invariante se 
$$ f(V_j) \subseteq V_j \quad \forall j =1, \dots, n$$
\end{defn}
\spazio
\begin{prop} Sia $f \in End(V) $.\\
I seguenti fatti sono equivalenza
\begin{itemize}
\item[(i)]$\exists \B$ base di $V$ tale che $M_\B(f)$ \'e triangolare superiore 
\item[(ii)]$P_f(t)$ \'e completamente fattorizzabile
\item[(iii)]$\exists \B $ base di $V$ tale che $ \mathfrak{F}_\B$ \'e $f$-invariante
\end{itemize}
\proof \bianco
\begin{itemize}
\item (i) $\implica $ (ii)\\
Dalla forma triangolare superiore della matrice otteniamo:
$$P_f(t)=( \mu_1 -t ) \cdots (\mu_n -t ) $$
\item (iii) $\implica $ (ii) \\
Dimostriamolo per induzione su $\dim V=n \geq 1 $\\
Per $n=1$ \'e ovvio\\
Supponiamo che valga per qualsiasi spazio vettoriale di dimensione $n$.\\
Sia $W$ un sottospazio di dimensione $n+1$ e sia $f\in End(W)$.\\
Ora poich\'e il polinomio caratteristico di $f$ \'e completamente fattorizzabile,$$\exists \mu \in Sp(f)\quad \implica \quad \exists v \in V , \, v\neq 0 \tc f(v)=\mu v $$
Sia $V$ tale che 
$$ W = Span(v) \oplus V $$ e sia $\D=\{ v \} \cup \B$ una base adattala alla decomposizione, ne segue che 

$$ A=M_\D(f) =   
\left(
\begin{array}{c|c}
\mu_1 & \star \\
\hline
0 & M_\B(f_{\vert V})
\end{array}
\right)
$$
Ora poich\'e $\dim V = n $ e il polinomio caratteristico della restrizione \'e fattorizzabile, posso concludere con l'ipotesi induttiva
\item (i) $\ses $ (iii) \\
Sia $\B =\{ \nvett \} $ tale che $ \mathfrak{F}_\B$ \'e $f$-invariante allora
$$f(V_1)\in V_1 \implica f(v_1)=\mu_1 v_1 $$
$$f(V_2)\in V_2 \implica f(v_2)=\star v_1 + \mu_2 v_2$$
$$f(V_3)\in V_3 \implica f(v_3)=\star v_1 + \star v_2 \mu_3 v_3 $$
da cui procedendo per induzione otteniamo 
$$ M_\B(f) = \left( \begin{array}{cc} \begin{array}{cc} \mu_1 &  \\ & \mu_2 \end{array} & \star \\  0 &  \begin{array}{cc} \ddots &  \\ & \mu_n \end{array} \end{array}\right)
$$
Allo stesso modo il viceversa


\end{itemize}
\end{prop}
\begin{defn} Se $f\in End(V) $ verifica queste condizioni \'e detto triangolabile
\end{defn}

\begin{cor}Se $\K$ \'e algebricamente chiuso allora tutti gli endomorfismi di $V$ sono triangolabili
\end{cor}
\newpage
\subsubsection{Simultanea triangolazione}
\begin{defn}$f,g\in End(V)$ si dicono simultaneamente triangolabili se esiste una base di $V$ a bandiera sia per $f$ che per $g$
\end{defn}
\spazio 
\begin{lem}Sia $f\in End(V)$ triangolabile e sia $W\subseteq V $ $f$-invariante.\\
Allora $f_{\vert W } $ \'e triangolabile
\proof Sia $\D$ una base di $W$ estendiamola a $\B$ base di $V$.\\
Allora chiamando $A=M_\D(f_{\vert W } $ otteniamo
$$ B = \left( \begin{array}{c|c} 
A & C \\
\hline 
0 & D 
\end{array} \right)$$
Ora $$p_f(t) = \det ( B - TI)= \det (A-tI) \det ( D-tI) = p_A(t) \cdot q(t) $$
ed essendo $f$ triangolabile, $p_f$ completamente fattorizzabile e dunque anche $p_A$ lo \'e, consegue che $f_{\vert W } $ \'e triangolabile
\end{lem}
\begin{prop}Siano $f,g \in End(V) $ triangolabili tali che $f\circ g = g \circ f $.\\
Allora
\begin{itemize}
\item[(i)] $f$ e $g$ ammettono un autovalore comune
\item[(ii)]$f$ e $g$ sono simultaneamente triangolabili

\end{itemize}
\proof \bianco
\begin{itemize}
\item[(i)] $f$ \'e triangolabile dunque $\exists V_\lambda \neq \{ 0 \} $\\
Ora $\forall v \in V_\lambda $ 
$$ f( g(v)) = g(f(v))= \lambda (g(v)) \quad \implica \quad g(v) \in V_\lambda $$
e visto che vale $\forall v \in V_\lambda $ allora $V_\lambda $ \'e $g$-invariante.\\
Per il lemma precedente, allora, $ g_{\vert V_\lambda } $ \'e triangolabile dunque 
$$ \exists v \in V_\lambda \text{ autovettore per } g_{\vert V_\lambda} \quad \implica \quad v \text{ autovettore per } g $$
Ora $v\in V_\lambda$ quindi \'e autovettore sia per $f$ che per $g$ 
\end{itemize}
\end{prop}
\newpage
\section{Endomorfismi coniugati}

\begin{defn}[Coniugati]\bianco
Siano $f,g \in End(V) $, $f$ e $ g$ si dicono coniugati $ f\sim g $ se 
$$ \exists h \in GL(V) \quad g = h^{-1} \circ f \circ h $$ 
\end{defn} \spazio
Visto in versione matriciale
\begin{defn}[Simili]\bianco
Siano $A,B \in M(n,\K) $, $A$ e $B$ si dicono simili $ A \sim B $ se 
$$ \exists M \in GL(n,\K) \quad B=M^{-1} \cdot A \cdot M $$
\end{defn} \spazio
\begin{oss} Per alleggerire la notazione poniamo 
$$M_\B=M_\B^\B$$
\end{oss}
\begin{prop} I seguenti fatti sono equivalenti
\begin{itemize}
\item[(i)] $ f\sim g$ 
\item[(ii)] $\forall \B $ base di $V$, $M_\B(f) \sim M_\B(g)$
\item[(iii)] $\exists \B, \D $ basi di V tale che $ M_\B(f)=M_\D(g)$
\end{itemize}
Questi fatti si possono dimostrare in modo analogo alle dimostrazioni fatte per  SD-equivalenza
\end{prop}
\spazio
Mostriamo quali sono gli invarianti per la relazione studiata
\begin{prop}[Invarianti] \bianco
La lista degli invarianti per coniugio e similitudine sono
\begin{itemize}
\item[(i)] Polinomio caratteristico 
\begin{itemize}
\item[(ii)] Spettro di $f$   
\item[(iii)] Determinante di $f$
\end{itemize}
\item[(iv)] Molteplicit\'a geometrica
\begin{itemize}
\item[(v)] Rango di $f$
\end{itemize}

\end{itemize}
\proof Mostriamo che valgono le seguenti implicazioni
\begin{itemize}
\item (i) $ \implica $ (ii) \\
Se $P_f(t)=P_g(t)$ allora in particolare i 2 polinomi avranno in comune le radici che altro non sono che gli elementi dello spettro
\item (i) $\implica $ (iii)\\
Dalla definizione di polinomio caratteristico segue che 
$$P_A(t) = \det ( A -tI) $$ quindi
$$ \det A = P_A(0) $$
\item (iv) $\implica$ (v) \\
Infatti da 
$$V_{\lambda}= \ker (f - \lambda id ) $$ 
segue che
$$ \ker f = V_0$$ 
quindi posso ricavarmi il rango di $f$ dalla dimensione degli autospazi usando la formula per nucleo e immagine
$$ \dim Im(f)=n-\dim \ker f= n-\dim V_0 = n -d_0 $$
\end{itemize}
Da quanto detto sopra basta dimostrare solamente (i) e (iv)
\begin{itemize}
\item[(i)]Se $B\sim A $ allora $B= PAP^{-1} $ dunque
$$ B-\lambda I = PAP^{-1} - \lambda I = PAP^{-1} - \lambda PIP^{-1} = P ( A - \lambda I ) P^{-1} $$
Ora calcolando il determinante e usando la formula di Binet
$$ P_B(t) = \det P \cdot P_A(t) \det P ^{-1} $$ 
Ma essendo $\K$ un campo, vale la propiet\'a commutativa dunque
$$ P_B(t)=P_A(t) $$
\item[(iv)] Se $h \sim h' $\\

Essendo i 2 endomorfismi simili vale che $\exists \beta \in GL(W) $ tale che
$$ h' = \beta \circ h \circ \beta^{-1} $$
$\forall v \in \ker h $ 
$$ h'(\beta (v)) = ( \beta \circ h\circ \beta^{-1})(\beta (v))= ( \beta \circ h\circ \beta^{-1} \circ \beta )(v) = \beta(h(v)=\beta(0)=0$$
Quindi $\beta (\ker h \subseteq \ker h')$, in modo analogo si mostra che vale l'altra inclusione.\\
Essendo i 2 nuclei isomorfi, hanno la stessa dimensione

\end{itemize}
\endproof
\end{prop}
\begin{oss} Il set di invarianti a nostra disposizione non \'e completo , infatti prendiamo come esempio
$$ A= \begin{pmatrix}
0 & 1 & 0 & 0 \\ 
0 & 0 & 1 & 0 \\ 
0 & 0 & 0 & 0 \\ 
0 & 0 & 0 & 0
\end{pmatrix}   \quad B= \begin{pmatrix}
0 & 1 & 0 & 0 \\ 
0 & 0 & 0 & 0 \\ 
0 & 0 & 0 & 1 \\ 
0 & 0 & 0 & 0
\end{pmatrix} $$ 
in 	questo caso il polinomio minimo \'e $t^4$.\\
Inoltre $V_0(A) =V_0(B)=2$ ma essendo $A^2 \neq 0 $ e $B^2 =0 $ non possono essere simili
\end{oss}
\spazio

\newpage
\subsection{Decomposizione primaria}
\begin{thm}[Decomposizione primaria]\bianco
Sia $ f \in End(V) $ e $ p(t) \in I(f) $\\
Se $$ p(t)=a(t) b(t) \text{ con } MCD(a(t), b(t))=1 $$
Allora $$V =\ker (a(f)) \oplus \ker (b(f))$$\\
inoltre entrambi gli addendi sono $f$-invarianti 
\proof Dall'identit\'a di Bezout segue che 
$$ 1=a(t) m(t) + b(t)n(t) $$ 
Ora valutando in $f$ ottengo
$$ id_V=a(f)\circ m(f)+ b(f) \circ n(f)$$
$\forall v \in V $ 
$$ v=(a(f)\circ m(f))(v)+ (b(f)\circ n(f))(v)$$ 
Osserviamo che $(a(f)\circ m(f)) \in \ker b(f) $ infatti
$$ (b(f) \circ a(f) \circ m(f)) (v)= m(f) \circ (p(f) (v)) =0 \in End(V) $$
dove il primo uguale, viene giustificato dall'osservazione~\ref{commu}.\\
In modo analogo si prova che $(b(f)\circ n(f)) \in  \ker a(f) $\\
Da ci\'o concludiamo che $$V =\ker (a(f)) + \ker (b(f))$$
Mostriamo che l'intersezione tra i 2 sottospazi \'e ridotta al solo $0$ \\
Sia $ w \in \ker (a(f)) \cap \ker (b(f))$ allora dalla formula precedente 
$$  w=(a(f)\circ m(f))(w)+ (b(f)\circ n(f))(w)=0$$ 
La $f$-invarianza \'e lasciata per esercizio
\endproof
\end{thm}

\newpage

\subsection{Caso triangolabile}
Restringiamoci al caso triangolabile  \\ \\
Essendo $f$ triangolabile 
$$ p_f(t) =\pm ( t-\lambda_1) ^{m_1} \cdots ( t-\lambda_k)^{m_k} $$
Iterando la decomposizione primaria otteniamo 
$$ V=\bigoplus_{j=1}^k \ker ( (t-\lambda_j \cdot Id)^{m_j} ) = \bigoplus_{j=1}^kW_j$$
con i $W_i$ $f$-invarianti\\
Chiamiamo $g_j$ la riduzione di $f$ a $W_j$ \\ \\
Sia $\{ \B_1 ,\, \cdots , \, \B_k \} $ una base di $V$ adattata alla decomposizione ovvero $\B_i $ \'e base di $W_j$
$$ M_\B(f) =\begin{pmatrix}
A_1 \\
 & A_2 \\
 & & \ddots \\
 & & & A_k 
\end{pmatrix}$$
dove ogni blocco ha come autovalore $\lambda_j$ infatti
$$ P_f(t)=P_{A_1}(t) \cdots  P_{A_k} (t) $$
quindi $\forall j $ vale
$$ P_{g_j} = \pm ( t-\lambda_j) ^{ m_j} $$
da cui $\dim W_j=m_j $

Ora come sappiamo il polinomio minimo ha le stesse radici del polinomio caratteristico dunque
$$ q_f(t) =\pm ( t-\lambda_1) ^{r_1} \cdots ( t-\lambda_k)^{r_k}  \quad \text{  con }  1\leq r_j \leq m_j $$
E considerando la restrizione
$$ q_{g_j}(t) =\pm ( t-\lambda_j) ^{\overline{r}_j} \quad \text{ con } 1 \leq \overline{r}_j \leq m_j$$
A priori  sappiamo che $r_j\geq \overline{r}_j $ invece 
\begin{prop} $\forall j \, r_j=\overline{r}_j $
\proof
Supponiamo che $ \overline{r}_1 < r_1 $ invece i restanti sono uguali da cui
$$ (t-\lambda_1 )^{\overline{r}_1 } (t -\lambda_2)^{r_2} \cdots ( t- \lambda_k ) ^{r_k} $$ tale polinomio appartiene all'ideale ma ha grado minore del polinomio minimo.\\
Assurdo dunque era assurda l'ipotesi $ \overline{r}_1 < r_1 $ , dunque vale l'uguaglianza
\endproof
\end{prop}
\newpage
Per quanto detto nella pagina precedente lo studio della relazione di similitudine nel caso di endomorfismi triangolabili, si riduce allo studio di endomorfismi "pi\'u semplici" con questa propriet\'a 
$$ g:\, W \to W  \quad con \dim W=m $$
$$ P_g(t)=\pm ( t-\lambda)^m $$
$$ q_g(t)=\pm (t-\lambda)^r$$
dove resta da studiare i vari casi al variare di  $r$ tra $1 \leq r \leq m $ \\ \\

Per comodit\'a di esposizione, restringiamoci al caso di endomorfismi nilpotenti ovvero

 \begin{defn}[Endomorfismo nilpotente]\bianco
 $g :\, W \to W $ si dice nilpotente se l'unico autovalore \'e 0 .\\
\end{defn}
\begin{oss}Una definizione equivalente di nilpotente.\\
$g $ \'e nilpotente se esiste $k\in \N $ tale che $g^k \equiv 0 $
\end{oss}
Dunque dopo aver applicato questa restrizione otteniamo
$$ P_g(t) = t^m $$ $$q_g(t) = t^r \quad 1\leq r \leq m  $$
\\
 Mostriamo che la riduzione \'e fittizia \\
Sia $g \in End(W) $ tale che $sp(g)=\{ \lambda \} $
$$ g= \lambda Id + (g -\lambda id) = \lambda id + h$$
dove $h=g-\lambda id_v$ viene chiamata \textbf{parte nilpotente }di $g$.\\
Mostriamo che appunto $h$ \'e nilpotente.\\
Essendo $g$ triangolabile allora esiste una base $\B$ di $W$ tale che
$$ M_\B = \begin{pmatrix}
\lambda & & \star \\ 
& \ddots & \\ 
0& & \lambda
\end{pmatrix} = \lambda I_m +  \begin{pmatrix}0 & & \star \\ 
& \ddots & \\ 
0& & 0
\end{pmatrix}  = M_\B(\lambda id_v) + M_\B(h)$$
La matrice che rappresenta $h$ \'e triangolare superiore e lungo la diagonale ha solo $0$ quindi $h$ \'e nilpotente 

\begin{prop} 2 endomorfismi sono simili se lo sono le loro classi nilpotenti
\proof
Supponiamo che $$ g=\lambda id + h $$
$$ g'=\lambda id + h' $$
Se $g \sim g' $ allora 
$$\exists \beta \in GL(W) \quad g'=\beta \circ g \circ \beta^{-1}$$
Quindi
$$ g' = \beta \circ ( \lambda id + h ) \circ \beta^{-1} = \lambda id + \beta \circ h \circ h^{-1}$$
Dunque $$h'= \beta \circ h \circ \beta^{-1} \quad \implica \quad h \sim h^{-1} $$
\end{prop} 
\newpage
Dopo queste restrizioni, non resta che studiare cosa succede al variare di $r$
\paragraph*{$r=1$}
$$q_h (t) =t $$ dunque $$q_h(f)= h =0 \in End(W) \quad \implica  \quad h \text{ diagonalizzabile}$$
\paragraph*{$r=m$}
$$P_h(t)=q_h(t) $$
Essendo il polinomio minimo di grado $m$  $$t^{m-1} \not in I(h) $$ 
quindi 
 $$h^{m-1} \neq 0  \quad \implica \quad \exists v \in W \tc h^{m-1} (v) \neq 0 $$
Consideriamo i seguenti vettori
 $$ v ,\, h(v) , \, \cdots , \, h^{m-1} (v) $$
 
\begin{lem}$\{ v ,\, h(v) , \, \cdots , \, h^{m-1} (v)  \} $ \'e una base di $W$ 
\proof Per quanto detto sopra i vettori sono tutti diversi da $0$, poich\'e sono $\dim W =m$ basta mostrare che sono linearmente indipendenti.\\
Sia 
$$ a_0 v+ a_1 h(v) +  \cdots + a_{m-1} h^{m-1} (v)=0$$
Ora se applichiamo $h$ alla combinazione lineare otteniamo 
$$ a_0 h(v)+ a_1 h^2(v) +  \cdots + a_{m-1} h^{m} (v)=0$$
Notiamo che l 'ultimo addendo \'e 0 infatti $h^m$ \'e l'endomorfismo nullo.\\
Iterando l'applicazione di $h$ alla combinazione  otteniamo 
$$a_0 h^{m-1} (v)=0 $$
e poich\'e $h^{m-1}(v) \neq 0 $ allora $a_0=0$ \\
Risalendo si dimostra per induzione che 
$$a_j =0 \quad \forall j=0, \cdots , m-1 $$ dunque i vettori sono linearmente indipendenti e formano una base di $W$ 
\endproof
\end{lem}
Tale base viene chiamata base ciclica di $h$ rispetto a $v$ 
 $$\B = \{ h^{m-1} (v) , h^{m-2}(v) , \cdots , \, v \} $$
tale che$$ M_\B(h) =
\begin{pmatrix}

\boldsymbol{0} & 1 & 0 & \cdots & 0 & 0 \\ 
0 & \boldsymbol{0} & 1 & \cdots & 0 & 0 \\ 
0 & 0 & \boldsymbol{0} & \cdots & 0 & 0 \\ 
\vdots & \vdots & \vdots & \vdots & \vdots & \vdots \\ 
0 & 0 & 0 & \cdots & \boldsymbol{0} & 1 \\ 
0 & 0 & 0 & \cdots & 0 & \boldsymbol{0}
\end{pmatrix} $$
Tale matrice che ha tutti $0$ sulla diagonale e 1 sulla sovra diagonale \'e detto \textbf{ blocco di Jordan } di autovalore 0 e taglia m e si indica con $J(0,m)$  \newpage
Possiamo riassumere quanto detto sopra con 
\begin{prop} Nel caso in cui $r=m$.\\
Siano $f,g$ due endomorfismi di $W$ nilpotenti allora
$$ f \sim g \quad \ses \quad \exists\B,\D \text{ basi di W } \tc M_\B(f) =M_\D(g)=J(0,m) $$
\end{prop}
\begin{oss} Nel caso in cui gli  endomorfismi non siano nilpotenti ovvero
$$ P_g(t)=q_g(t)=(t- \lambda)^m $$ 
Allora la forma matriciale normale \'e $J(\lambda,m) $ come nel caso nilpotente ma sulla diagonale invece di esserci $0$ ci sar\'a $\lambda $
\end{oss} \newpage

Il set di invarianti a nostra disposizione non \'e sufficiente per studiare il caso generale ovvero quando $0 < r < m $
\begin{lem}[Nuclei successivi]\bianco 
Sia $f\in End(V) $ allora
$$ \ker f^i = \ker f^{i+1} \quad \implica \quad \ker f^{i+1} = \ker f^{i+2}$$
Ovvero le dimensioni dei nuclei continuano a crescere finch\'e non se ne trovano 2 successivi con la stessa dimensione
\proof $\ker f^{i+1} \subseteq \ker f^{i+2} $ in modo evidente\\ \\
Mostriamo, dunque l'altra inclusione
$$ \forall v \in \ker f^{i+2}  \quad f^{i+2}(v) =0 \quad \implica \quad f^{i+1 } (f(v))=0 \quad \implica \quad f(v) \in \ker f^{i+1} $$ 
Ore per ipotesi $\ker f^{i+1} = \ker f^i$ quindi 
$$ f(v) \in \ker f^i \quad \implica f^i ( f(v))=0 \quad \implica \quad f^{i+1}(v)=0 \quad \implica \quad v\in \ker f^{i+1}$$
\endproof
\end{lem}

\spazio 
Consideriamo sempre 
$$ h:\, W \to W \quad Sp(h)=\{ 0\} $$ 
Allora possiamo considerare la stringa
$$d_1<\cdots < d_r=m \qquad \forall j \quad d_j=\dim \ker h^i $$ 

\begin{lem}[Stringa di dimensioni invariante]
Siano $h \sim h' $.\\
Allora $h$ e $h'$ hanno la stessa stringa di dimensioni
\proof
Essendo i 2 endomorfismi simili vale che $\exists \beta \in GL(W) $ tale che
$$ h' = \beta \circ h \circ \beta^{-1} $$
$\forall v \in \ker h $ 
$$ h'(\beta (v)) = ( \beta \circ h\circ \beta^{-1})(\beta (v))= ( \beta \circ h\circ \beta^{-1} \circ \beta )(v) = \beta(h(v)=\beta(0)=0$$
Quindi $\beta (\ker h )\subseteq \ker h'$, in modo analogo si mostra che vale l'altra inclusione.\\
Essendo i 2 nuclei isomorfi, hanno la stessa dimensione \endproof
\end{lem}
\begin{oss} Nel caso in cui $r=1$ la stringa era $(d_1=m)$ \\invece nel caso $r=m$ era $(1,2,\cdots, m ) $\end{oss} 
\newpage
\paragraph*{$1<r<m$}


\begin{defn}[Base di Jordan bene ordinata] \bianco
Una base di Jordan bene ordinate per h \'e una base $\B$ tale che $M_\B(h)$ 
\begin{itemize}
\item[(i)]\'e diagonale a blocchi
\item[(ii)] Ogni blocco lungo la diagonale \'e di Jordan $J(0,m_i) $
\item[(iii)] $m_i \leq m_{i-1} $ scendendo lungo la diagonale si trovano via via blocchi di taglia minore o uguale del blocco precedente
\end{itemize}
\end{defn}
\spazio
\begin{thm}[Forma normale di Jordan - caso nilpotente]\bianco
Per ogni $h$ endomorfismo nilpotente con stringa ($0<d_1< \cdots< d_r<m$) esiste una base $\B$ di Jordan bene ordinata per $h$ tale che la matrice in forma normale $M_\B(h)$ \'e completamente determinata dalla stringa
\proof
Consideriamo le inclusioni
$$\ker h \subset \ker h^2 \subset \cdots \subset \underbrace{ \ker h^{r-2}  \subset  \ker h^{r-1} \subset \ker h^r=W}_{\text{ soffermandoci su questo }} $$

Facciamo ora una costruzione che andr\'a reiterata.\\ \\ 
Essendo $\ker h^{r-1} $ un sottospazio di $W $ considero $U$ un complementare di $\ker h^{r-1}$ da cui
$$ W= \ker h^{r-1} \oplus U $$
Notiamo che la scelta di $U$ \'e arbitraria invece la sua dimensione \'e $t=d_r-d_{r-1}$.\\
Fissiamo una base  di U 
$$ u_1 , \cdots , u_r $$
a applichiamo a questi vettori $h$ ottenendo
$$ h(u_1),\cdots , h(u_t)$$
Mostriamo che 
\begin{itemize} \item questi vettori appartengono al $\ker h^{r-1} $ : 
$$ h^{r-1}(h(u_i))= h^r (u) =0 \text{ esendo il polinomio minimo di h di grado } r $$ \item
e sono linearmente indipendenti:\\
supponiamo che 
$$ a_1 h(u_1) + \cdots + a_t h(u_t)=0 $$
allora applicando $h^{r-2}$ e usando la inearit\'a di $h$ ottengo
$$h^{r-1} (  a_1 u_1  + \cdots+ a_t u_t)=0  $$
Ma $ a_1 u_1 + \cdots + a_t u_t \in \ker h^{-1} \cap U $ quindi essendo i 2 spazi in somma diretta  deve succedere che 
$$  a_1 u_1 + \cdots + a_t u_t =0 \quad \implica \quad a_1=\cdots=a_t=0 \quad\text{ i vettori } u_i \text{ formano una base } $$ 
\end{itemize}
\spazio
Ora mi sposto e considero 
$$ \ker h^{r-3} \subset \ker h^{r-2} \subset \ker ^{r-1} $$
e riapplicando la costruzione ottengo
$$ \ker h^{r-1} = \ker h^{r-2} \oplus U'$$
ma questa volta impongo che la base di $U'$ sia
$$ h(u_1) , \cdots , h(u_t) , u_{t+1}, \cdots , u_{t'} $$
il numero dei vettori da aggiungere \'e univocamente determinato dalla stringa.\\
Applico $h$ alla base di $U'$ ottenendo
$$ h^2(u_1) , \cdots , h^2(u_t) , h(u_{t+1}), \cdots ,h( u_{t'}) $$
Reitero finch\'e possibile \\
\\
Ho costruito una base di $W$ arrangiata in questa tabella, dove le altezze e le lunghezze dei gradini dipendono solamente dalla stringa\\
\begin{center}
\begin{tabular}{cccccc}

$u_1$ & $\cdots $ & $u_t$ &  &  &  \\ 

$h(u_1) $& $\cdots $&$ h(u_t) $& $u_{t+1}$ & $\cdots$ & $u_{t'} $\\ 

$ h^2(u_1)$ & $\cdots$ & $h^2(u_t)$ & $h(u_{t+1})$ & $\cdots$ & $h(u_{t'})$ \\ 
\end{tabular} 
\end{center}
Considerando le colonne$C_1, \cdots C_s$ della tabella ottengo che
$$ W = \bigoplus_{i=1}^n Span(C_i) $$
inoltre ogni $Span(C_i)$ \'e h-invariante ed i vettori di ogni colonna formano una base ciclica per h ristretta a questi $Span$ .\\
Ora se riordino le colonne al contrario(basso verso alto) ottengo da ogni colonna una base di Jordan , inoltre l'altezza di ogni colonna diminuisce andando verso destra quindi sono ben ordinate.
\endproof
\end{thm}
\begin{cor} La stringa \'e un invariante completo per la relazione di coniugio ristretta agli endomorfismi triangolarizzabili
\end{cor}

\spazio
\begin{thm}[Unicit\'a] \bianco
Siano $B$ e $B'$ due basi di Jordan bene ordinate per $h$ \\
Allora
$$M_\B (h) = M_{\B'}(h)$$ \\
Data una matrice di Jordan, posso calcolare la stringa invariante.
\end{thm} 
\spazio
\newpage


\subsection{Studio della coniugazione in $\R$}
Nel caso di endomorfismi triangolabili, abbiamo studiato la relazione in modo esauriente, avendo trovato per ogni endomorfismo un suo rappresentante in forma normale per la relazione di coniugazione.\\
In un campo non algebricamente chiuso le cose sono pi\'u difficili perch\'e non si conosce quali sono i polinomi irriducibili, fatta eccezione per $\R$

\subsubsection*{Complessificazione}
Mostriamo un esempio di complessificazione che ci permetter\'a di trovare gli elementi irriducibile dell'anno dei polinomi a coefficienti reali.
\paragraph{Irriducibili in $\R[t]$}
Su $\C$ \'e definita la seguente funzione chiamata coniugio
$$ \C \to \C \quad a+ib \to a+i(-b) = a-ib $$
con le seguenti propiet\'a
\begin{itemize}
\item[(i)] $ \overline{z+w}=\overline{z}+ \overline{w}$
\item[(ii)]$ \overline{z\cdot w}=\overline{z} \cdot  \overline{w}$
\item[(iii)]  $\frac{ z +  \overline{z}}{2}=  Re z $
\item[(iv)] $ \frac{z - z }{2 i }=Im z $
\item[(v)] $ z \cdot \overline{z} = \vert z \vert ^2 =Re z ^2 + Im z ^2$
\end{itemize}
Inoltre possiamo definire l'insieme dei numeri reali come
\begin{itemize}
 \item[(vi)] $\R=\{ z \in \C \, \vert z = \overline{z } \} $

\end{itemize}

Ora siccome $\R \subset \C $ allora in particolare vale $\R[t] \subset \C[t]$\\
Sia $p[t] \in \R[t] $ che abbia una radice non reale $\alpha $ quindi
 $$p(\alpha)=0 \quad \implica \quad \overline{p(\alpha)}= \overline{0}$$ 
Ora utilizzando le propiet\'a (i) (ii) e (vi) otteniamo 
$$ P(\overline{\alpha})=0 $$
Da questo fatto segue che 
$$ p(t) = (t- \lambda_1)^{m_1} \cdots (t-\lambda_s)^{m_s} (t-\alpha_1)^{l_1} (t - \overline{\alpha_1})^{l_1} \cdots \quad \text{ in } \C $$
dove $\lambda_i \in \R $ 
invece $\alpha_i \neq \overline{\alpha_1}$
Ora consideriamo  con $\alpha_1 =a_1 + i b_1$
$$ Q_{\alpha_1}(t) = ( t- \alpha_1 ) (t- \overline{\alpha_1} ) = t^2 + ( \alpha_1+ \overline{ \alpha_1}) t  + \alpha_1 \overline{\alpha_1}$$
Ora $( \alpha_1+ \overline{ \alpha_1})  = 2 a_1  \in \R $ e $  \alpha_1 \overline{\alpha_1} = a_1^2 + b_1^2 \in R$\\ 
Da quanto detto sopra  gli irriducibili in $\R[t]$ sono i polinomi di primo grado e quelli di secondo grado che non hanno radici reali 
\newpage

\subsubsection*{Complessificazione di uno spazio vettoriale}
Sia $V$ uno spazio vettoriale su $\R$.\\
Definiamo $V_\C $ (complessificato di $V$) come la coppia ordinata $ (v , w) \in V \times V $.\\
Per assonanza con i numeri complessi denotiamo la coppia $(v,w) = v +i w $\\
\\
Muniamo $V_\C $ di 2 operazioni $+ , \cdot $ che lo rendano uno spazio vettoriale su  $\C$ 
$$ + : \, V_\C \to V_\C \qquad  ( v, w) + (v',w') \to (v+v, w+w') $$
$$ \cdot: \, \C \times V_\C \qquad (\alpha ,(v,w)) \to ( av -bw, aw + bv) \quad \text{ supponendo } \alpha=a+ ib $$
\begin{prop} $V_\C$ con le 2 operazioni definite \'e uno spazio vettoriale su $\C$.\end{prop}
Definiamo anche su $V_\C$ un applicazione coniugio
$$ V_\C \to V_\C \qquad (v,w) \to (v,-w) $$
inoltre $V \subset V_\C $ infatti $V=\{z \in V_\C \, \vert \, z =\overline{z}\}$
\spazio
\begin{lem}[Base reale]\bianco
Se $\B=\{ \nvett \} $ \'e una base di $V$.\\
Allora $\B$ \'e anche base di $V_\C$ detta base rea;e di $V_\C$ 
\proof $ \forall z =(v,w) \in V_\C $ 
$$ z  =(a_1 v_1 + \cdots + a_n v_n ) + i ( b_1 v_1 + \cdots + b_n v_n) $$ 
infatti $v,w \in V$.\\
Per quanto detto sopra $\B$ genera $V_\C$ \\ \\ 
Mostriamo che i vettori sono linearmente indipendenti su $\C$ 
$$ \alpha_1 v_1 + \cdots + \alpha_n v_n =0+i0 \qquad \text{ con } \alpha_i=a_i + i b_i  \in \C$$
Ora applicando la definizione di somma e prodotto in $V_\C$ otteniamo
$$ a_1 v_1 + \cdots + a_n v_n + i( b_1 v_1 + \cdots + b_n v_n )=0+0i$$
Dunque abbiamo 2 combinazioni reali nulle di vettori di una base dunque
$$ a_1= \cdots = a_n = b_1= \cdots = b_n =0 $$
I vettori di $\B$ sono linearmente indipendenti su $\C$ 
\endproof
\paragraph*{Applicazione lineare complessificata}
$$\begin{tikzcd} 
V \arrow{r}{f} \arrow[d,hook] 
&W \arrow[d,hook]\\ 
V_\C \arrow[dashed]{r}{f_\C}&W_\C
\end{tikzcd} $$
\end{lem}
Dove $f_\C$ \'e definita in modo che sia $\C$-lineare e renda commutativo il diagramma quindi
$$ f_\C ((v,w))=(f(v),f(w))$$
Se fissiamo una base $\B$ di $V$ ed una base $\D$ di $W$ allora
$$ \AB (f) =\AB (f_\C) \quad \implica \quad
P_f(t) = P_{f_\C} (t) \in \R[t]$$ 
\newpage
\paragraph{Relazione di coniugazione su $\R$}
Possiamo ora studiare la relazione di coniugio su $End(V) $ e su $End(V_\C) $\\
$$\R[t] \ni P_f(t) = \cdots (t-\lambda)^{m} \cdots Q_\alpha (t)^{l}(t) \cdots $$
Ora se guardiamo il polinomio in $\C[t] $ allora
$$ \C[t] \ni P_{f_\C}(t) = \cdots (t-\lambda)^{m} \cdots (t-\alpha)^l
( t-\overline{\alpha})^l \cdots$$
Possiamo applicare ad entrambi le fattorizzazioni la decomposizioni primaria
$$\text{ su } \R \qquad  V = \cdots \oplus \ker (f-\lambda id )^m  \oplus \cdots \oplus \ker Q_{\alpha} ^l(f) \oplus \cdots $$
$$\text{ su } \C \qquad  V_\C = \cdots \oplus \ker (f_\C -\lambda id )^m  \oplus \cdots \oplus \ker (f_\C - \alpha id)  ^l \oplus (f_\C - \overline{\alpha} id )^l \oplus \cdots $$
 \begin{prop} Valgono i seguenti fatti
 \begin{itemize}
 \item[(i)] $\ker (f_\C - \lambda id ) $ \'e il complessificato di $\ker (f - \lambda id ) $
 \item[(ii)] $\ker  (f_\C - \alpha id )\oplus \ker (f_\C - \overline{\alpha} id ) $  \'e il complessificato di $\ker Q_\alpha(f) $ 
 \end{itemize}
 \end{prop}
 Dal fatto (i) riesco a trovare una base di Jordan reale che sar\'a anche una base di Jordan complesso.\\
 \\
Dal fatto (ii) posso prendere una base $\B$ di Jordan per $\ker \ker (f_\C - \alpha id )$ e poi prendere $\overline{\B}$ come base dell'altro sottospazio e poi considerare il cambiamento di base
$$ \B , \overline{\B} \Longleftrightarrow Re \B , Im B $$
La seconda base \'e chiamata base di Jordan reale.\\
I
L'applicazione $f$ ristretta a $\ker Q_\alpha(f) $ rappresentata tramite la base $ \B , \overline{\B} $ \'e della forma
$$ \begin{pmatrix}
J(\alpha,s) & 0 \\
0 & J(\overline{\alpha},s) & 0 
\end{pmatrix} \in M(2s ,\C) 
$$
Invece tramite la base $\Re \B, \Im \B$ 
$$ \left( \begin{array}{c| c | c |c }
A & I_2 & 0 &0 \\
\hline
0 & \ddots & \ddots & 0 \\ 
\hline
0 & 0 & \ddots & I_2 \\
\hline
0 & 0 & 0 & A 

\end{array} \right)  $$
dove  
$$ A = \begin{pmatrix}
\Re ( \alpha ) & \Im(\alpha)\\
-\Im (\alpha ) & \Re (\alpha ) 
\end{pmatrix}$$

\newpage
\subsection{Calcolo della forma di Jordan}
Un modo per calcolare facilmente la forma di Jordan \'e conoscere la dimensione dei nuclei successivi, questa proposizione \'e molto utile
\begin{prop} 
$$ \dim \ker f^k- \dim \ker f^{k-1} \quad \text{ \'e decrescente} $$
\proof
$$ f^ k = f \circ f^{k-1}$$ da questo segue che 
$$ \dim \ker f^k = \dim \ker f^{k-1} + \dim ( \Im f^{k-1} \cup \ker f ) $$
Questa successione \'e decrescente perch\'e la dimensione delle immagine \'e decrescente 
\endproof
\end{prop}
Inoltre valgono le seguenti considerazioni 
\begin{itemize}

\item $m_a(\lambda_i) $ 
\begin{itemize}
\item nel polinomio caratteristico indica la somma delle taglie dei blocchi relativi a $\lambda_i$
\item nel polinomio minimo indica la taglia massima dei blocchi relativi a $\lambda_i$
\end{itemize}
\item $m_g(\lambda_i)$ indica il numero di blocchi relativi a $\lambda_i$
\item $\ker (f -\lambda_i id )^\beta $ \'e generato dai primi $\beta $ vettori di ogni blocco relativo a $\lambda_i$
\item numero di blocchi relativi a $\lambda_i $ di taglia $\alpha$
$$ 2 \dim \ker (( f-\lambda_i Id) ^\alpha) - \dim \ker (( f-\lambda_i Id) ^{\alpha+1}) - \dim \ker (( f-\lambda_i Id) ^{\alpha-1})$$  
\end{itemize}
\newpage
\section{Complementi}

\subsubsection{Centro degli endomorfismi}
\begin{lem} $f\in End(V) $ $$ \forall v \in V \, v \neq 0 \quad v \text{ \'e autovettore per } f \ses f \in Span(id_v) $$
\proof $\implica$ Sia $\B = \{ \nvett \} $ \'e una base di $V$ , quindi \'e fatto di autovettori
$$ M_\B(f) = \left( \begin{array}{cc} \begin{array}{cc} \lambda_1 &  \\ & \lambda_2 \end{array} & \star \\  0 &  \begin{array}{cc} \ddots &  \\ & \lambda_n \end{array} \end{array}\right)
$$
Dobbiamo mostrare che tutti i $\lambda_i$ siano uguali a $\lambda_1$\\
$$\forall i=2,\cdots, n  \quad v_1+v_i  \text{ \'e un autovettori quindi }\exists \lambda \in \K \tc f(v_1+v_i)=\lambda(v_1+v_i)$$

$$  f(v_1+v_i) = \lambda_1 v_1 + \lambda_i v_i = \lambda v_1 + \lambda v_i$$
$$ v_1 ( \lambda_1 - \lambda ) + v_i ( \lambda_i -\lambda ) =0  $$
Ora $v_1$ e $v_1$ appartengono ad una base quindi sono linearmente indipendenti da cui $\lambda=\lambda_1=\lambda_i $\\
L'altra freccia \'e ovvia
\endproof
\end{lem}
\spazio
\begin{prop}[Centro delle matrici]
Sia $ A \in M(n,\K) $ 
$$ \forall B \in M(n,\K) \quad AB=BA \implica A \in Span(I_nI$$
\proof
Basta vedere che tutti gli elementi di $\K^n $ sono autovettori per $A$ poi applicare il lemma\\
Sia $v \in \K^n $ $ v\neq 0 $ e completiamo a $\B =\{ v, \, v_2, \, \cdots , \, v_n \} $ base di $\K^n $ \\
Costruiamo l'applicazione relativa a B , in modo che $v$ sia l'unico autovettore relativo ad $1$
$$ v \to v $$
$$ v_i \to 0 \quad \forall i=2,\cdots , n $$
Per come abbiamo costruito B \'e diagonalizzabile con spettro $\{ 1, 0 \} $\\
Ora $AB=BA$ quindi per il lemma iniziale $V_1(B) $ \'e $A$-invariante 
$$ Av \in Span(v) \quad Av=\lambda v $$ 
dunque v \'e autovettore per $A$
\endproof
\end{prop}
%\end{document}
